\PassOptionsToPackage{dvipsnames}{xcolor}
\documentclass[ignorenonframetext, hyperref=unicode,compress]{beamer}


\usepackage{cmap}
\usepackage[T2A]{fontenc}
\usepackage[utf8x]{inputenc}
\usepackage[bulgarian]{babel}
\selectlanguage{bulgarian}

\usepackage{graphicx}
\usepackage{listings}

\hypersetup{
	colorlinks=true,
	linkcolor=blue,
	filecolor=blue,
	urlcolor=blue,
	anchorcolor=blue,
	citecolor=blue
}

\lstset{language=java, 
  numbers=left, 
  numberstyle=\tiny,
  stepnumber=1, 
  numbersep=3pt, 
  tabsize=2, 
  texcl,
  basicstyle=\ttfamily\small,
  identifierstyle=\ttfamily\small,
  keywordstyle=\sffamily\bfseries\small,
  extendedchars=true, inputencoding=utf8x,
  backgroundcolor=\color[rgb]{1,1,0.845},
  escapeinside={/*@}{@*/}}

\newcommand{\Cpp}{{\ttfamily\bfseries C++}}
\newcommand{\CC}{{\ttfamily\bfseries C}}

% \usepackage{algpseudocode}

%\usepackage{ucs}

%\logo{\includegraphics[height=0.4cm]{../macros/logo_elsys.png}}

\titlegraphic{\href{http://creativecommons.org/licenses/by-sa/3.0/}{\includegraphics{pics/cc.png}}}

\newcommand{\lubo}{%
\author[Л.~Чорбаджиев]{Ненко Табаков, Пламен Танов, Любомир Чорбаджиев}
\institute[ELSYS] {%
Технологично училище ``Електронни системи'' \\
Технически университет, София
}}

\usepackage{tikz}
\usetikzlibrary{arrows,shapes.misc,calc,backgrounds,fit,decorations.pathmorphing,positioning,chains,scopes,topaths,automata,shapes,trees}

\tikzset{obj/.style={rectangle,minimum size=6mm,
				very thick,draw=red!50!black!50,top color=white,bottom color=red!50!black!20, font=\ttfamily},
ref/.style={rectangle,
		minimum size=8mm,
		% The rest
		very thick,draw=black!50,
		top color=white,bottom color=black!20},
	every node/.style={font=\ttfamily},
	>=stealth',
	thick,
	draw=black!50,
	shorten >=1pt,node distance=2cm and 2.5cm,on grid}


\mode<article>
{

}

\mode<presentation>
{
  \usetheme[secheader=true]{Madrid}
  \usecolortheme{crane}
  \usefonttheme[onlylarge]{structurebold}
  \setbeamercovered{transparent}
}

\usepackage[unicode]{hyperref}

%%% Local Variables: 
%%% mode: latex
%%% TeX-master: t
%%% End: 


\title{Нишки в Java}
\authors
\date{\today}


\begin{document}



\frame{\titlepage}

\begin{frame}
\small
{\bf Забележка:} Тази лекция е адаптация на лекции от следните
курсове:
\begin{itemize}
  \item 
Prof. Judson Harward, Prof. Steven Lerman:
\href{http://ocw.mit.edu/OcwWeb/Civil-and-Environmental-Engineering/1-00Fall-2005/CourseHome/index.htm}{
{\em 1.00 / 1.001 Introduction to Computers and Engineering Problem Solving  Fall 2005} (MIT OpenCourseWare:
Massachusetts Institute of Technology)}\\
{\bf Лиценз:}
\href{http://ocw.mit.edu/OcwWeb/web/terms/terms/index.htm\#cc}{Creative commons
BY-NC-SA}  
\item Dr. George Kocur: 
\href{http://ocw.mit.edu/OcwWeb/Civil-and-Environmental-Engineering/1-00Spring-2005/CourseHome/index.htm}{
{\em 1.00 Introduction to Computers and Engineering Problem
Solving  Spring 2005} (MIT OpenCourseWare: Massachusetts Institute of
Technology)} \\
{\bf Лиценз:}
\href{http://ocw.mit.edu/OcwWeb/web/terms/terms/index.htm\#cc}{Creative commons
BY-NC-SA}
\end{itemize}

\end{frame}

\begin{frame}
{\bf Литература:} 
\begin{itemize}
\item
\href{http://java.sun.com/docs/books/tutorial/essential/concurrency/index.html}{The
Java Tutorials. Trail: Essential Classes; Lesson: Concurrency.}
\end{itemize}
\end{frame}

\begin{frame}
\frametitle{Съдържание}
\tableofcontents %[hideallsubsections]
\end{frame}


\section{Процеси и нишки}
%---------------------------------------------------------------------- SLIDE -
\begin{frame}[containsverbatim]
\frametitle{Процеси и нишки}
\begin{itemize}
\item Повечето операционни системи позволяват едновременно (конкурентно)
изплънение на няколко процеса.
\item Процесите са ресурсоемки.
\item Процесите са толкова добре изолирани един от друг, че това прави
комуникацията между тях трудна и  сложна.
\item За разлика от процесите нишките са значително по-леки и по-малко
ресурсоемки. Нишките работят в рамките на едни процес, и между тях на практика
няма изолация.
\end{itemize}
\end{frame}
%------------------------------------------------------------------------------

\selection{Нишки в Java}
%---------------------------------------------------------------------- SLIDE -
\begin{frame}[containsverbatim]
\frametitle{Нишки в Java}
\begin{itemize}
\item Java е един от малкото езици за програмиране, които имат вградена
в езика поддръжка за многонишково програмиране.
\item Един от първите масови езици за програмиране,  които разполагат с
вградена поддръжка за многонишково програмиране е Ada.
\item C\# също притежва вградена поддръжка за многонишково програмиране.
\item Езиците {\CC} и {\Cpp} нямат вградена поддръжка за многонишково
програмиране. Вместо това те разчитат на външни библиотеки, които да
предоставят такава поддръжка.
\item Очакваният \href{http://en.wikipedia.org/wiki/C\%2B\%2B0x}{нов стандарт за
\Cpp} предвижда въвеждане в стандартната \Cpp\ библиотека на средства за
многонишково програмиране (%\url{http://www.open-std.org/jtc1/sc22/wg21/},
\url{http://www.open-std.org/jtc1/sc22/wg21/docs/papers/2008/n2497.html}).
\end{itemize}
\end{frame}
%------------------------------------------------------------------------------

%---------------------------------------------------------------------- SLIDE -
\begin{frame}[containsverbatim]
\frametitle{Нишки в Java}
\begin{itemize}
\item В Java сибирането на ``боклука'' от обекти, които не са нужни, се
изпълнява от виртуалната машина в отделна нишка.
\item Библиотеките за създаване на графични интерфейси използват отделна нишка
за обработка и разпределение на събитията, генерирани от графичния интерфейс.
Това позволява на графичния интерфейс на приложението да остава отзивчив дори
когато приложението е заето с дълги пресмятания или входно/изходни операции.
\item Всичко това показва, че на езика Java му е вътрешно присъща
могонишковост. Дори когато дадена програма е написана така, че да не използва
повече от една нишки, средата в която се изпълнява тази програма използва много
нишки.
\item Java предоставя средства, чрез които програмистите могат лесно да създават
многонишкови приложения.
\end{itemize}
\end{frame}
%------------------------------------------------------------------------------

\section{Класът \lstinline{Thread}}
%---------------------------------------------------------------------- SLIDE -
\begin{frame}[containsverbatim]
\frametitle{Класът \lstinline{Thread}}
\begin{itemize}
\item Класът \lstinline{Thread} предоставя поддръжка за създаване на
нишки в Java.
\item Има два начина, по които можете да кажете на нишката какво да прави:
\begin{itemize}
  \item Като наследим класа \lstinline{Thread} и заместим метода
  \lstinline{run()}.
  \item Като предадем на конструктора на класа \lstinline{Thread} обект, който
  реализира интерфейса \lstinline{Runnable}.
\end{itemize}
\end{itemize}
\end{frame}
%------------------------------------------------------------------------------


%---------------------------------------------------------------------- SLIDE -
\begin{frame}[containsverbatim]
\frametitle{Класът \lstinline{Thread}: заместване на метода \lstinline{run()}}
\begin{itemize}
\item Наследяваме класа \lstinline{Thread} и предефинираме метода
\lstinline{void run()}:
\begin{lstlisting}
class MyThread extends Thread {
	...
	void run() {
		// код, който се изпълнява от нишката
	}
}
\end{lstlisting}

\item За да създадем нишка трябва да създадем обект от класа
\lstinline{MyThread} и да извикаме метода \lstinline{start()}.
\begin{lstlisting}
MyThread myThread=new MyThread();
myThread.start();
\end{lstlisting}
\end{itemize}
\end{frame}
%------------------------------------------------------------------------------

%---------------------------------------------------------------------- SLIDE -
\begin{frame}[containsverbatim]
\frametitle{Класът \lstinline{Thread}: използване на \lstinline{Runnable} }
\begin{itemize}
\item Друг възможен вариант за описание на поведението на нишка е да се
използва отделен клас, който наследява интерфейсът \lstinline{Runnable}:
\begin{lstlisting}
interface Runnable {
	public void run();
}
\end{lstlisting}
\item За тази цел трябва де дефинирате клас, който реализира интерфейса
\lstinline{Runnable}:
\begin{lstlisting}
class MyRunnable implements Runnable {
	public void run{
		//  код, който се изпълнява в нишка
	}
}
\end{lstlisting}
\end{itemize}
\end{frame}
%------------------------------------------------------------------------------


%---------------------------------------------------------------------- SLIDE -
\begin{frame}[containsverbatim]
\frametitle{Класът \lstinline{Thread}: използване на \lstinline{Runnable} }
\begin{itemize}
\item При създаването на нишка в конструктора на класа \lstinline{Thread} се
предава обект от класа \lstinline{MyRunnable}:
\begin{lstlisting}
Thread myThread=new Thread(new MyRunnable());
myThread.start();
\end{lstlisting}
\item Една от причините да се използва този подход при дефиниране на
поведението на нишки е, че в Java класовете могат да наследяват само един клас.
При това положение, ако класът в който трябва да се опише поведение на нишка,
вече е наследил друг клас, то използването на предходната стратегия става
невъзможна.
\end{itemize}
\end{frame}
%------------------------------------------------------------------------------

\section{Пускане и спиране на нишки}
%---------------------------------------------------------------------- SLIDE -
\begin{frame}[containsverbatim]
\frametitle{Пускане и спиране на нишки}
\begin{itemize}
\item За да се пусне нишка е необходимо да се създаде обект от класа
\lstinline{Thread} или негов наследник и да се извика метода
\lstinline{start()}.
\begin{lstlisting}
Thread t1=new MyThread();
t1.start();
Thread t2=new Thread(new MyRunnable());
t2.start();
\end{lstlisting}
\item Класът \lstinline{Thread} по принцип притежва метод
\href{http://java.sun.com/javase/6/docs/api/java/lang/Thread.html#stop()}{\lstinline{stop()}}.
Целта на този метод е била той да се използва за спиране не нишки. Този метод
обаче е несигурен от гледна точка на конкурентното програмиране. Поради това
изпозването на този метод не се препоръчва и предстои 
\href{http://java.sun.com/javase/6/docs/technotes/guides/concurrency/threadPrimitiveDeprecation.html}{махането}
на този метод от класа \lstinline{Thread}.
\end{itemize}
\end{frame}
%------------------------------------------------------------------------------

%---------------------------------------------------------------------- SLIDE -
\begin{frame}[containsverbatim]
\frametitle{Пускане и спиране на нишки}
\begin{itemize}
\item Начинът за спиране на една нишка е нейният метод \lstinline{run()} да
завърши работа. Не използвайте метода \lstinline{stop()} за да спрете нишка.
\item Класът \lstinline{Thread} притежава метод \lstinline{isAlive()}, с който
можете да проверите дали нишката все още работи или е завършила работа:
\begin{lstlisting}
Thread t=new MyThread();
t.start();
...
if(t.isAlive()) {
	// нишката все още работи
} else {
	// нишката е завършила
}
\end{lstlisting}
\end{itemize}
\end{frame}
%------------------------------------------------------------------------------

%---------------------------------------------------------------------- SLIDE -
\begin{frame}[containsverbatim]
\frametitle{Пускане и спиране на нишки}
\begin{itemize}
\item Класът \lstinline{Thread} притежава метод \lstinline{join()}, с чиято
помощ можете да изчакате завършването на работата на дадена нишка:
\begin{lstlisting}
Thread t=new MyThread();
t.start();
...
t.join(); // спира и изчаква докато нишката t завърши
...
\end{lstlisting}
\end{itemize}
\end{frame}
%------------------------------------------------------------------------------

\section{Пример за използване на нишки}
%---------------------------------------------------------------------- SLIDE -
\begin{frame}[containsverbatim]
\frametitle{Пример за използване на нишки}
\begin{itemize}
\item 
\end{itemize}
\end{frame}
%------------------------------------------------------------------------------

%---------------------------------------------------------------------- SLIDE -
\begin{frame}[containsverbatim,shrink=5]
\frametitle{Пример за използване на нишки}
\lstinputlisting[firstline=1,lastline=14]{src/labs/threads/URLCopyMain.java}
\end{frame}
%------------------------------------------------------------------------------

%---------------------------------------------------------------------- SLIDE -
\begin{frame}[containsverbatim,shrink=5]
\frametitle{Пример за използване на нишки}
\lstinputlisting[firstline=16,lastline=130,firstnumber=last]{src/labs/threads/URLCopyMain.java}
\end{frame}
%------------------------------------------------------------------------------


%---------------------------------------------------------------------- SLIDE -
\begin{frame}[containsverbatim,shrink=5]
\frametitle{Пример за използване на нишки}
\lstinputlisting[firstline=1,lastline=14]{src/labs/threads/URLCopyThread.java}
\end{frame}
%------------------------------------------------------------------------------

%---------------------------------------------------------------------- SLIDE -
\begin{frame}[containsverbatim,shrink=5]
%\frametitle{Пример за използване на нишки}
\lstinputlisting[firstline=16,lastline=32,firstnumber=last]{src/labs/threads/URLCopyThread.java}
\end{frame}
%------------------------------------------------------------------------------

%---------------------------------------------------------------------- SLIDE -
\begin{frame}[containsverbatim,shrink=5]
%\frametitle{Пример за използване на нишки}
\lstinputlisting[firstline=33,firstnumber=last]{src/labs/threads/URLCopyThread.java}
\end{frame}
%------------------------------------------------------------------------------

\section{Необходимост от синхронизация на нишки}
%---------------------------------------------------------------------- SLIDE -
\begin{frame}[containsverbatim,shrink=5]
\frametitle{Необходимост от синхронизация на нишки}
\lstinputlisting[firstline=1,lastline=19]{src/labs/threads/UnsynchornizedExample.java}
\end{frame}
%------------------------------------------------------------------------------

%---------------------------------------------------------------------- SLIDE -
\begin{frame}[containsverbatim,shrink=5]
%\frametitle{Пример за използване на нишки}
\lstinputlisting[firstline=21,lastline=150,firstnumber=last]{src/labs/threads/UnsynchornizedExample.java}
\end{frame}
%------------------------------------------------------------------------------

%---------------------------------------------------------------------- SLIDE -
\begin{frame}[containsverbatim,shrink=5]
\frametitle{Необходимост от синхронизация на нишки}
\begin{itemize}
\item Дадени са 50 нишки, които инкрементират една и съща променлива 1000000
пъти. Очакването е след като нишките прюключат работа, стойността на
променливата \lstinline{value} да бъде равна на \lstinline{50*1000000=50000000}.
\item Резултата от няколко последователни пускания на програмата е следния:
\begin{verbatim}
value=47643339
value=47822482
value=50000000
value=48498127
\end{verbatim}
\item Проблемът е липсата на синхронизация на нишките при едновременен достъп
до общ ресурс --- променливата \lstinline{value}.
\end{itemize}
\end{frame}
%------------------------------------------------------------------------------

%---------------------------------------------------------------------- SLIDE -
\begin{frame}[containsverbatim]
\frametitle{Атомарност}
\begin{itemize}
\item За да бъде коректно предходната програма
\lstinline{UnsynchronizedExample.java}, е необходимо операцията:
\begin{lstlisting}[numbers=none] 
value++;
\end{lstlisting}
да бъде атомарна.
\item {\em Атомарна} се нарича операция, която се изпълнява изцяло, без да бъде
прекъсвана от виртуалната машина.
\end{itemize}
\end{frame}

%---------------------------------------------------------------------- SLIDE -
\begin{frame}[containsverbatim]
\frametitle{Атомарност}
\begin{itemize}
  \item Операцията
  \begin{lstlisting}[numbers=none]
	value++;
  \end{lstlisting}
  типично не е атомарна.
  \item Изпълнението на операцията \lstinline{value++} може да се представи по
  следния начин:
\begin{lstlisting}
	local1=value;
	local1=local1+1;
	value=local1;
\end{lstlisting}
\end{itemize}
\end{frame}

\section{Състояние на надпревара}
%---------------------------------------------------------------------- SLIDE -
\begin{frame}[containsverbatim]
\frametitle{Състояние на надпревара}
\begin{itemize}
  \item Когато две нишки или повече нишки се опитат конкурентно да променят
  състоянието на общ ресурс, инструкциите им могат да се преплетат.
  \item Преплитането на инструкциите на две или повече нишки зависи от
  начина, по който нишките се планират върху процесора.
\end{itemize}
\end{frame}

%---------------------------------------------------------------------- SLIDE -
\begin{frame}[containsverbatim]
\frametitle{Състояние на надпревара}
\begin{itemize}
\item Да предположим, че първоначално \lstinline{value=5}. Един вариант за
преплитане на инструкциите на две нишки е следния:
\begin{lstlisting}
thread[0]: local0=value       /*local1 = 5*/
thread[0]: local0=local0+1    /*local1 = 6*/
thread[1]: local1=value       /*local2 = 5*/
thread[1]: local1=local1+1    /*local2 = 6*/
thread[1]: value=local1       /*value = 6 */
thread[0]: value=local0       /*value = 6 */
\end{lstlisting}
\item Стойността на \lstinline{value} става 6, докато правилния
резултат трябва да бъде 7.
\end{itemize}
\end{frame}

%---------------------------------------------------------------------- SLIDE -
\begin{frame}[containsverbatim]
\frametitle{Състояние на надпревара}
\begin{itemize}
  \item {\em Състояние на надпревара (race condition)} се нарича ситуацията, при която
  няколко
  процеса или нишки конкурентно манипулират данни, които са общи, поделени между
  конкурентните процеси (нишки).
  \item Крайната стойност на поделените (общите) данни зависи от начина, по
  който процесите се планират върху процесора.
  \item За да се предотврати състоянието на надпревара (race condition),
  манипулацията на общите ресурси от конкурентните процеси (нишки) трябва да
  бъде синхронизирана.
\end{itemize}
\end{frame}

%-------------------------------------------------------------------- SECTION -
\section{Критичен участък (critical section)}

%---------------------------------------------------------------------- SLIDE -
\begin{frame}[containsverbatim]
\frametitle{Критичен участък (critical section)}
\begin{itemize}
  \item Няколко процеса или нишки се състезават за достъп до общи (поделени)
  данни.
  \item Всеки процес или нишка има участък от кода, в който работи с общите
  данни. Такъв участък от кода се нарича {\em критичен участък}. 
  \item Трябва да се изгради механизъм, чрез който да се гарантира, че когато
  един процес се намира в критична секция, никой друг процес не може да
  влезе в своя критичен участък.
\end{itemize}
\end{frame}


\section{Критичен участък в Java: \lstinline{sychronized}}
%---------------------------------------------------------------------- SLIDE -
\begin{frame}[containsverbatim]
\frametitle{Критичен участък в Java: \lstinline{sychronized}}
\begin{itemize}
\item За дефиниране на критични участъци от
кода Java предоставя ключовата дума \lstinline{synchronized}.
\item Ключовата дума \lstinline{synchronized} може да се използва по няколко
начина. Една от възможните употреби е методи да се дефинират като
\lstinline{synchronized}.
\begin{lstlisting}
class Foo {
...
	public synchronized void bar() {
		...
	} 
...
}
\end{lstlisting}
\end{itemize}
\end{frame}

%---------------------------------------------------------------------- SLIDE -
\begin{frame}[containsverbatim]
\frametitle{Критичен участък в Java: \lstinline{sychronized}}
\begin{itemize}
\item Дефинирането на метода \lstinline{bar()} като \lstinline{synchronized}
означава, че този метод не може да бъде прекъснат (не може да се преплете) с
други \lstinline{synchronized} методи, работещи върху същия обект.
\item Ако друга нишка се опита да изпълни \lstinline{synchronized} метод
върху същия обект, то тя ще бъде блокирана, докато първия
\lstinline{synchronized} метод не завърши работа.
\item Обърнете внимание, че \lstinline{synchronized} методи се блокират само от 
други \lstinline{synchronized} методи. Методите, които не са синхронизирани,
работят независимо.
\item Синхронизират се методите, работещи върху един и същи обект. Докато
работи \lstinline{synchronized} метод върху даден обект, друга нишка може да
изпълнява \lstinline{synchronized} метод на друг обект.
\end{itemize}
\end{frame}

%---------------------------------------------------------------------- SLIDE -
\begin{frame}[containsverbatim]
\frametitle{Критичен участък в Java: \lstinline{sychronized}}
\begin{itemize}
\item 
\end{itemize}
\end{frame}


\end{document}
