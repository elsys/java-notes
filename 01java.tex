\PassOptionsToPackage{dvipsnames}{xcolor}
\PassOptionsToPackage{unicode}{hyperref}
\documentclass[ignorenonframetext, hyperref=unicode,unicode]{beamer}


\usepackage{cmap}
\usepackage[T2A]{fontenc}
\usepackage[utf8x]{inputenc}
\usepackage[bulgarian]{babel}
\selectlanguage{bulgarian}

\usepackage{graphicx}
\usepackage{listings}

\hypersetup{
	colorlinks=true,
	linkcolor=blue,
	filecolor=blue,
	urlcolor=blue,
	anchorcolor=blue,
	citecolor=blue
}

\lstset{language=java, 
  numbers=left, 
  numberstyle=\tiny,
  stepnumber=1, 
  numbersep=3pt, 
  tabsize=2, 
  texcl,
  basicstyle=\ttfamily\small,
  identifierstyle=\ttfamily\small,
  keywordstyle=\sffamily\bfseries\small,
  extendedchars=true, inputencoding=utf8x,
  backgroundcolor=\color[rgb]{1,1,0.845},
  escapeinside={/*@}{@*/}}

\newcommand{\Cpp}{{\ttfamily\bfseries C++}}
\newcommand{\CC}{{\ttfamily\bfseries C}}

% \usepackage{algpseudocode}

%\usepackage{ucs}

%\logo{\includegraphics[height=0.4cm]{../macros/logo_elsys.png}}

\titlegraphic{\href{http://creativecommons.org/licenses/by-sa/3.0/}{\includegraphics{pics/cc.png}}}

\newcommand{\lubo}{%
\author[Л.~Чорбаджиев]{Ненко Табаков, Пламен Танов, Любомир Чорбаджиев}
\institute[ELSYS] {%
Технологично училище ``Електронни системи'' \\
Технически университет, София
}}

\usepackage{tikz}
\usetikzlibrary{arrows,shapes.misc,calc,backgrounds,fit,decorations.pathmorphing,positioning,chains,scopes,topaths,automata,shapes,trees}

\tikzset{obj/.style={rectangle,minimum size=6mm,
				very thick,draw=red!50!black!50,top color=white,bottom color=red!50!black!20, font=\ttfamily},
ref/.style={rectangle,
		minimum size=8mm,
		% The rest
		very thick,draw=black!50,
		top color=white,bottom color=black!20},
	every node/.style={font=\ttfamily},
	>=stealth',
	thick,
	draw=black!50,
	shorten >=1pt,node distance=2cm and 2.5cm,on grid}


\mode<article>
{

}

\mode<presentation>
{
  \usetheme[secheader=true]{Madrid}
  \usecolortheme{crane}
  \usefonttheme[onlylarge]{structurebold}
  \setbeamercovered{transparent}
}

\usepackage[unicode]{hyperref}

%%% Local Variables: 
%%% mode: latex
%%% TeX-master: t
%%% End: 

\title{Въведение в Java}
\authors
\date{\today}

\pdfinfo{
	/Title Introduction to Java
}

\begin{document}



\frame{\titlepage}

\begin{frame}


\small
{\bf Забележка:} Тази лекция е адаптация на:
\begin{itemize}
  \item \href{http://ocw.mit.edu/NR/rdonlyres/Electrical-Engineering-and-Computer-Science/6-092January--IAP--2006/2C5A832A-E1BA-4104-A129-CC998C7B95FC/0/lecture1a.pdf}{Lucy Mendel: {\em Introduction and Java Programming}} from
\href{http://ocw.mit.edu/OcwWeb/Electrical-Engineering-and-Computer-Science/6-092January--IAP--2006/CourseHome/index.htm}{
{\em 6.092: Java for 6.170} (MIT OpenCourseWare:
Massachusetts Institute of Technology)}\\
{\bf Лиценз:}
\href{http://ocw.mit.edu/OcwWeb/web/terms/terms/index.htm\#cc}{Creative commons
BY-NC-SA}  

\end{itemize}


\end{frame}

\begin{frame}
\frametitle{Съдържание}
\tableofcontents %[hideallsubsections]
\end{frame}


\begin{frame}
 \frametitle{Литература}
\begin{itemize}
\item
\href{http://computing.open.ac.uk/m254/}{Java Everyware}
\item
\href{http://java.sun.com/docs/books/tutorial/index.html}{The
Java Tutorials.}
\item
\href{http://java.sun.com/developer/onlineTraining/Programming/BasicJava1/}{Essentials of the Java Programming Language, Part 1}

\end{itemize}
\end{frame}

\begin{frame}
 \frametitle{Какво е Java}
\begin{itemize}
\item Java е обектно–ориентиран език за програмиране
\item В Java всичко (или по-точно -- почти всичко) е обект -- обектът има данни и операции, които може да извършва над тях
\item Обектът е инстанция на клас -- всеки обект има тип
\item В Java има и примитивни типове -- \lstinline{int},  \lstinline{double}, \lstinline{long},...
\end{itemize}
\end{frame}


\begin{frame}[containsverbatim]
\frametitle{Hello Java World!}
 \begin{lstlisting}
package  hello;

public  class HelloWorld {

 	String myString;

 	void shout() {
 		myString = "Hello Java World!";
 		System.out.println(myString);
	}

 	public static void main(String[] args) {
 		HelloWorld myHelloWorld = new HelloWorld();
		myHelloWorld.shout();
	}
}
\end{lstlisting}
\end{frame}

\begin{frame}[containsverbatim]
\frametitle{Класове}
 \begin{itemize}
 \item Класовете могат да се разглеждат като спецификация, описание на обектите
\item Структурата и поведението на всеки обект се описва от неговия клас
\end{itemize}
\begin{lstlisting}
 class Point{
	private double x;
	private double y;
	...
	public void add(Point p);
	...
}
\end{lstlisting}
\end{frame}


\begin{frame}[containsverbatim]
\frametitle{Полете (член-променливи)}
 \begin{itemize}
 \item Полетата (атрибути, член-променливи) определят състоянието на класа
\end{itemize}
\begin{lstlisting}
class HelloWorld {
	private String myString;
	...
}
\end{lstlisting}
\begin{lstlisting}
class Point {
	public double x;
	public double y;
	...
}
\end{lstlisting}
\end{frame}


\begin{frame}[containsverbatim]
\frametitle{Обекти}
 \begin{itemize}
 \item Обектът са инстанция на клас
 \item Всеки обект има състояние, поведение и идентичност.
 \item За създаването на обект трябва да се използва операторът \lstinline{new}.
\end{itemize}
\begin{lstlisting}
class HelloWorld {
	String myString;
	...

	public static void main(String[] args) {
 		HelloWorld myHelloWorld = new HelloWorld();
	}
}
\end{lstlisting}
\end{frame}

\begin{frame}[containsverbatim]\frametitle{Използване на обекти}
\begin{lstlisting}
class ShowPoint {

	public static void main(String[] args) {
		Point myPoint = new Point ();
		myPoint.x = 10.0;
		myPoint.y = 15.0;
	
		System.out.println(myPoint.x);
		System.out.println(myPoint.y);
	}
}
\end{lstlisting}
\end{frame}

\begin{frame}[containsverbatim]\frametitle{Примитивни типове}
\begin{itemize}
\item В Java освен обекти има и примитивни типове: \lstinline{boolean}, \lstinline{byte},\lstinline{short},\lstinline{int},\lstinline{long},\lstinline{double},\lstinline{float}, \lstinline{char}.
\item За създаване на екземпляр (инстанция) на примитивен тип не е необходимо използването на оператора \lstinline{new}.
\end{itemize}

\begin{lstlisting}
class ShowPoint {

	public static void main(String[] args) {
		Point myPoint = new Point ();
		myPoint.x = 10.0;
		myPoint.y = 15.0;
	
		System.out.println(myPoint.x);
		System.out.println(myPoint.y);
	}
}
\end{lstlisting}
\end{frame}

\begin{frame}[containsverbatim]\frametitle{Методи}
\begin{itemize}
\item Методите дефинират поведението на обектите.
\end{itemize}

\begin{lstlisting}
public  class HelloWorld {
 	String myString;

 	void shout() {
 		myString = "Hello, World!";
 		System.out.println(myString);
	}

 	public static void main(String[] args) {
 		HelloWorld myHelloWorld = new HelloWorld();
		myHelloWorld.shout();
	}
}
\end{lstlisting}
\end{frame}

\begin{frame}[containsverbatim]\frametitle{Тяло на метод}
\begin{itemize}
\item Методите могат да получават произволен брой аргументи
\item В тялото на метода могат да се дефинират локални променливи
\item Методите могат да връщат резултат (или да не връщата, 
ако типа на резултата е \lstinline{void}). Като резултат може да се върне 
само един обект.
\end{itemize}

\begin{lstlisting}
String firstname(String fullname) {
	int  space = fullname.indexOf(" ");
	String word = fullname.substring(0, space);
	return word;
}
\end{lstlisting}
\end{frame}



\begin{frame}[containsverbatim]\frametitle{Конструктори}
Конструкторите са специални методи
\begin{itemize}
 \item Не връщат стойност
 \item Използват се за първоначална инициализация
 \item Могат да имат параметри, както и нормално тяло, но оператора \lstinline{return}
 не може да се използва за връщане на стойност
\end{itemize}
\end{frame}

\begin{frame}[containsverbatim]\frametitle{Конструктори}
\begin{lstlisting}
public  class HelloWorld {
 	private String myString;
	
 	public HelloWorld (String helloMessage) {
 		myString = helloMessage;
	}
	
 	public HelloWorld () {
 		myString = "Hello, Java World";
	}
	
 	public void shout() {
 		System.out.println(myString);
	}
}\end{lstlisting}
\end{frame}

\begin{frame}[containsverbatim]\frametitle{Управление на хода на програмата}
\begin{columns}
 \column{0.45\textwidth}
\begin{lstlisting}
if(lucy.age < 18){
	// направи нещо
} else if(lucy.hasCar()){
 	// направи нещо друго
} else {
	// направи нещо трето
}
\end{lstlisting}
\column{0.45\textwidth}
\begin{lstlisting}
if ( cond1 ) {
	...
} else if ( cond2 ) {
 	...
} else if ( condN ) {
 	...
}
else {
	...
} 
\end{lstlisting}
\end{columns}
\end{frame}

\begin{frame}[containsverbatim]\frametitle{Условия и условни изрази}
\begin{itemize}
 \item Условие – израз, чиито резултат е истина или лъжа (\lstinline{true}, \lstinline{false})
\item Условни оператори: \lstinline{<}, \lstinline{>}, \lstinline{<=}, \lstinline{>=}, \lstinline{==}, \lstinline{!=}
\item Логически оператори: \lstinline{&&}, \lstinline{||}, \lstinline{!}
\end{itemize}
\begin{lstlisting}
box.isEmpty()
box.numberOfBooks()==0
!(box.numberOfBooks() > 1)
box.numberOfBooks() != MAX_NUMBER_OF_BOOKS 

lucy.age>= 21 && lucy.hasCar()
!someone.name.equals("Lucy"))
(!true || false) && true
\end{lstlisting}
\end{frame}

\begin{frame}[containsverbatim]\frametitle{Масиви}
\begin{itemize}
 \item Масивите са специални обекти
 \item Всеки масив има поле length, в което е записана големината на масива
\end{itemize}
\begin{lstlisting}
String[] pets = new String[2]; 
pets[0] = "Fluffy"; 
pets[1] = "Muffy"; 
		
String[]  otherPets = new String[] {"Fluffy", "Muffy"}; 
System.out.println(otherPets.length);
\end{lstlisting}
\end{frame}


\begin{frame}[containsverbatim]\frametitle{Цикъл \lstinline{for}}
\begin{itemize}
 \item Инициализация -- изпълнява се само веднъж
 \item Условие -- цикъла се изпълнява докато условието се изчислява като \lstinline{true}
 \item Стъпка -- изпълнява се всеки път
\end{itemize}
\begin{lstlisting}
for  (int i = 0; i<3; i++) {
	System.out.println(i);
}
\end{lstlisting}
\end{frame}


\begin{frame}[containsverbatim]\frametitle{Цикъл \lstinline{for} -- обхождане на колекция}
\begin{lstlisting}
String strArray[]=new String[10];
for(int i=0;i<strArray.length;i++){
	strArray[i]="Hello!"+i;
}

for(String str:strArray) {
	System.out.println(str);			
}
\end{lstlisting}
\end{frame}

\begin{frame}[containsverbatim]\frametitle{Цикъл \lstinline{while}}
\begin{itemize}
 \item Инициализация -- изпълнява се само веднъж
 \item Условие -- цикъла се изпълнява докато условието се изчислява като \lstinline{true}
 \item Стъпка -- изпълнява се всеки път
\end{itemize}
\begin{lstlisting}
for  (int i = 0; i<3; i++) {
	System.out.println(i);
}
\end{lstlisting}
\end{frame}


\end{document}
