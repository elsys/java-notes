\PassOptionsToPackage{dvipsnames}{xcolor}
\documentclass[ignorenonframetext, hyperref=unicode,compress]{beamer}


\usepackage{cmap}
\usepackage[T2A]{fontenc}
\usepackage[utf8x]{inputenc}
\usepackage[bulgarian]{babel}
\selectlanguage{bulgarian}

\usepackage{graphicx}
\usepackage{listings}

\hypersetup{
	colorlinks=true,
	linkcolor=blue,
	filecolor=blue,
	urlcolor=blue,
	anchorcolor=blue,
	citecolor=blue
}

\lstset{language=java, 
  numbers=left, 
  numberstyle=\tiny,
  stepnumber=1, 
  numbersep=3pt, 
  tabsize=2, 
  texcl,
  basicstyle=\ttfamily\small,
  identifierstyle=\ttfamily\small,
  keywordstyle=\sffamily\bfseries\small,
  extendedchars=true, inputencoding=utf8x,
  backgroundcolor=\color[rgb]{1,1,0.845},
  escapeinside={/*@}{@*/}}

\newcommand{\Cpp}{{\ttfamily\bfseries C++}}
\newcommand{\CC}{{\ttfamily\bfseries C}}

% \usepackage{algpseudocode}

%\usepackage{ucs}

%\logo{\includegraphics[height=0.4cm]{../macros/logo_elsys.png}}

\titlegraphic{\href{http://creativecommons.org/licenses/by-sa/3.0/}{\includegraphics{pics/cc.png}}}

\newcommand{\lubo}{%
\author[Л.~Чорбаджиев]{Ненко Табаков, Пламен Танов, Любомир Чорбаджиев}
\institute[ELSYS] {%
Технологично училище ``Електронни системи'' \\
Технически университет, София
}}

\usepackage{tikz}
\usetikzlibrary{arrows,shapes.misc,calc,backgrounds,fit,decorations.pathmorphing,positioning,chains,scopes,topaths,automata,shapes,trees}

\tikzset{obj/.style={rectangle,minimum size=6mm,
				very thick,draw=red!50!black!50,top color=white,bottom color=red!50!black!20, font=\ttfamily},
ref/.style={rectangle,
		minimum size=8mm,
		% The rest
		very thick,draw=black!50,
		top color=white,bottom color=black!20},
	every node/.style={font=\ttfamily},
	>=stealth',
	thick,
	draw=black!50,
	shorten >=1pt,node distance=2cm and 2.5cm,on grid}


\mode<article>
{

}

\mode<presentation>
{
  \usetheme[secheader=true]{Madrid}
  \usecolortheme{crane}
  \usefonttheme[onlylarge]{structurebold}
  \setbeamercovered{transparent}
}

\usepackage[unicode]{hyperref}

%%% Local Variables: 
%%% mode: latex
%%% TeX-master: t
%%% End: 


\title{Класове и интерфейси}
\authors
\date{\today}

\usetikzlibrary{shapes}

\begin{document}

\frame{\titlepage}

\begin{frame}
\small
{\bf Забележка:} Тази лекция е адаптация на:
\begin{itemize}
  \item \href{http://ocw.mit.edu/NR/rdonlyres/Electrical-Engineering-and-Computer-Science/6-092January--IAP--2006/66BAC837-433E-48A5-BA15-B766E0B7CDEA/0/lecture2a.pdf}{Justin Mazzola Paluska: {\em Classes and Interfaces}} from
\href{http://ocw.mit.edu/OcwWeb/Electrical-Engineering-and-Computer-Science/6-092January--IAP--2006/CourseHome/index.htm}{
{\em 6.092: Java for 6.170} (MIT OpenCourseWare:
Massachusetts Institute of Technology)}\\
{\bf Лиценз:}
\href{http://ocw.mit.edu/OcwWeb/web/terms/terms/index.htm\#cc}{Creative commons
BY-NC-SA}  

\end{itemize}

\end{frame}
\begin{frame}
\frametitle{Съдържание}
\tableofcontents %[hideallsubsections]
\end{frame}



\begin{frame}[containsverbatim]\frametitle{Терминология}
\begin{itemize}
 \item \lstinline{class} --- класовете могат да се разглеждат като описание, спецификация на обектите от дадения клас.
 \item \lstinline{interface} --- интерфейсът дефинира списъкът на достъпните методи
 \item {\bfseries instance} --- физическото представяне на даден клас или интерфейс в паметта
\end{itemize}

\begin{itemize}
 \item {\bfseries метод} --- функция, която е дефинирана в рамките на класа
 \item {\bfseries поле} --- променлива, която е част от класа
 \item {\bfseries статично поле} --- променлива, която е една и съща за всички инстанции на класа
\end{itemize}

\end{frame}


\begin{frame}[containsverbatim]\frametitle{Типове данни в Java}
В Java има два основни класа от типове данни:
\begin{itemize}
 \item примитивни типове
 \item обекти
\end{itemize}

\end{frame}


\begin{frame}[containsverbatim]\frametitle{Обекти}
\begin{itemize}
 \item Обектът е инстанция (екземпляр) на даден клас. Всеки обект принадлежи на даден клас.
 \item За създаване на инстанция на клас се използва операторът \lstinline{new}
 \item Операторът \lstinline{new}:
 \begin{itemize}
  \item Заделя място в паметта за новия обект
  \item Извиква съответния конструктор и инициализира обекта
  \item Връща препратка към новосъздадения обект
 \end{itemize}
\end{itemize}

\begin{lstlisting}
 Bean bean=new Bean();
\end{lstlisting}
\end{frame}

\begin{frame}[containsverbatim]\frametitle{Използване на обекти}
\begin{itemize}
 \item 
Обектите позволяват извикване на техните методи:
\begin{lstlisting}
public static void main(String[] args) {
	Bean bean = new Bean();
	bean.plantBean(); // Invoked on instance
}
\end{lstlisting}
\item
Обектите позволяват достъп до техните полета:
\begin{lstlisting}
public static void main(String[] args) {
		Point myPoint = new Point ();
		myPoint.x = 10;
		myPoint.y = 15;
}
\end{lstlisting}
\item Когато даден обект повече не ни трябва просто спираме да го използваме. Паметта заемана от него ще бъде освободена, когато няма повече препратки към него.
\end{itemize}

\end{frame}


\begin{frame}[containsverbatim]\frametitle{Дефиниране на клас}
\begin{itemize}
 \item 
Структурата на кода, необходим за дефиниране на клас, е следната:
\begin{lstlisting}
[<access>] [abstract/final] class <class_name>
		[extends <class_name>]
		[implements <interface_name>, ...]{
	//полета
	//конструктор
	//методи
}
\end{lstlisting}
\item Пример:
\begin{lstlisting}
public class Point {
	...
}
\end{lstlisting}
\end{itemize}
\end{frame}


\begin{frame}[containsverbatim]\frametitle{Членове на класа}
В рамаките на класа могат да се дефинират следните видове членове на класа:
\begin{itemize}
 \item конструктори
 \item полета (нестатични и статични)
 \item методи (нестатични и статични)
 \item вложени класове
\end{itemize}
\end{frame}

\begin{frame}[containsverbatim]\frametitle{Конструктор}
\begin{itemize}
 \item Конструкторът трябва да има същото име, като това на класа
 \item Един клас може да има няколко конструктора
 \item В конструктора се извършва инициализация на конструирания обект
\end{itemize}
\begin{lstlisting}
[<access>] class_name ([<argument list>]){
	//тяло на конструктора
}

public class Point {
	public Point () {
		...
	}
	...
}
\end{lstlisting}
\end{frame}


\begin{frame}[containsverbatim]\frametitle{Пример: Конструктори}
\begin{lstlisting}
public  class HelloWorld {
	
 	public HelloWorld (String helloMessage) {
 		myString = helloMessage;
	}
	
 	public HelloWorld () {
 		myString = "Hello, World";
	}
	
	public static void main(String[] args) {
 		HelloWorld myHelloWorld=new HelloWorld();
		HelloWorld myHelloWorld2=new HelloWorld("Hello!!!");
	}
}
\end{lstlisting}
\end{frame}

\begin{frame}[containsverbatim]\frametitle{Методи}
\begin{itemize}
 \item Методите описват поведението на класа. 
Те описват реакцията на обектите от класа на външни въздействия (съобщения). 
Методите извършват операции
 \item Методите работят върху състоянието (полетата) на класа
 \item Методите могат да имат произволен брой агрументи и могат връщат не повече от една стойност
 \item Ако даден метод не връща стойност, то неговият тип е void
 \item Един клас може да има произволен брой методи
\end{itemize}
\begin{lstlisting}
[<access>] <result type> method_name ([<argument list>]){
	//тяло на метода
}
\end{lstlisting}
\end{frame}

\begin{frame}[containsverbatim]\frametitle{Пример: Методи}
\begin{lstlisting}
class Box {
	public boolean isEmpty() {
		...
	}

	public int numberOfBooks() {
		...
	}
}
\end{lstlisting}
\end{frame}


\begin{frame}[containsverbatim]\frametitle{Предефениране на методи (overloading)}
\begin{itemize}
 \item В един клас може да има два (или повече) метода с еднакво име стига аргументите да са им различни
 \item При извикване методът се избира на основата на името и списъка от реално предадените аргументи
\end{itemize}
\begin{lstlisting}
void  foo () {
	...
}
void  foo (int a) {
	...
}
public  static void main (String[] args) {
 	obj.foo(); //извиква първия метод
 	obj.foo(7); //извиква втория метод
}
\end{lstlisting}
\end{frame}


\begin{frame}[containsverbatim]\frametitle{Полета}
\begin{itemize}
 \item Полето е част от клас
 \item Полето е променлива – съдържа данни
 \item Всяко поле има тип, който определя какъв вид данни ще се записват в него
\end{itemize}
\begin{lstlisting}
public  class Bean {
 	public int beanCounter = 0;
 	public Date date;

}
\end{lstlisting}
\end{frame}

\begin{frame}[containsverbatim]\frametitle{Пример}
\begin{lstlisting}
public class BankAccount {
 	private int balance;

 	public BankAccount() {
 		balance = 0;
	}

 	public void withdraw(int amount) {
 		balance = balance - amount;
	}

 	public void deposit(int amount) {
 		balance = balance + amount;
	}
}
\end{lstlisting}
\end{frame}

\begin{frame}[containsverbatim]\frametitle{Достъп до членовете на класа}
Достъпът до член на класа се дефинира с една от следните ключови думи:
\begin{itemize}
 \item \lstinline{public} -- всеки клас от всеки пакет има достъп
 \item \lstinline{protected} -- всеки подклас (наследник) има достъп
 \item \lstinline{(default)} -- само класове от същия пакет имат достъп
 \item \lstinline{private} -- само съответния клас има достъп
\end{itemize}
\end{frame}

\begin{frame}[containsverbatim]\frametitle{Наследяване}
\begin{itemize}
 \item Чрез наследяване даден клас може да придобие функционалност от друг клас
 \item Чрез наследяване може да се постигне абстракция на функционалността и данните
 \item Наследяването позволява да се намали сложността на големи софтуерни системи
\end{itemize}
\end{frame}

\begin{frame}[containsverbatim]\frametitle{Наследяване}
\begin{center}
\begin{tikzpicture}[grow=down,
    level 1/.style={sibling distance=4.5cm,level distance=3cm},
    level 2/.style={sibling distance=4.5cm, level distance=3cm},
    edge from parent/.style={very thick,draw=blue!40!black!60,
        shorten >=5pt, shorten <=5pt,arrows=triangle 90-},
    edge from parent path={(\tikzparentnode.south) -- (\tikzchildnode.north)},
    every node/.style={text ragged, inner sep=2mm},
    punkt/.style={rectangle, rounded corners, shade, top color=white,
    bottom color=blue!50!black!20, draw=blue!40!black!60, very thick}
    ]

\node[punkt, rectangle split, rectangle split parts=2] {BankAccount\nodepart{second}}
    %Lower part lv1
    child {
        node[punkt, rectangle split, rectangle split parts=2] {CheckingAccount
                  \nodepart{second}}
        edge from parent
            node {}
    }
    %Upper part, lv1
    child {
        node[punkt, rectangle split, rectangle split, rectangle split parts=2] {SavingsAccount
                  \nodepart{second}}
    };
\end{tikzpicture}
\end{center}

\begin{itemize}
 \item Две отделни идеи с различно поведение, но имат базова функционалност която е обща
\end{itemize}
\end{frame}

\begin{frame}[containsverbatim]\frametitle{Интерфейси}
\begin{itemize}
 \item Интерфейсът дефинира списък на достъпните методи
 \item В интерефейс се декларират методи, но не се дефинират
 \item Интерфейсите нямат конструктори
\end{itemize}
\begin{lstlisting}
interface BankAccount {
 	public void withdraw(int amount);
 	public void deposit(int amount);
}
\end{lstlisting}
\end{frame}

\begin{frame}[containsverbatim]\frametitle{Използване на интерфейси}
\begin{itemize}
 \item Един клас може да имплементира един или няколко интерфейса
 \item Един интерфейс може да разшири друг интерфейс
 \item Ако един клас имплементира даден интерфейс, то този клас трябва да 
предостави реализация на всеки метод от интерфейса
 \item Ако един клас имплементира няколко интерфейса то този клас трябва да 
предостави реализация на всеки метод от всеки интерфейс
\end{itemize}
\end{frame}

\begin{frame}[containsverbatim]\frametitle{Пример: Употреба на интерфейси}
\begin{lstlisting}
public  class CheckingAccount implements BankAccount {
	private int balance;

 	public CheckingAccount(int initial) {
 		balance = initial;
	} 
 	// implemented methods from BankAccount
 	public void withdraw(int amount) {
 		balance = balance - amount;
	}
 	public void deposit(int amount) {
 		balance = balance + amount;
	}
 	public int getBalance() {
 		return balance;
	}
}
\end{lstlisting}
\end{frame}

\begin{frame}[containsverbatim]\frametitle{Абстрактни класове}
\begin{itemize}
 \item Абстрактният клас е нещо средно между интерфейс и клас
\begin{itemize}
 \item може да има дефинирани методи
 \item може да има полета
 \item не могат да се създават обекти
\end{itemize}
 \item Помага да се дефинира една идея както като функционалност така и като данни
 \item В един абстрактен клас може да се разположат методи, които имат обща функционалност за всички подкласове
 \item Абстрактен клас се дефинира с ключовата дума abstract
\end{itemize}
\end{frame}

\begin{frame}[containsverbatim]\frametitle{Пример: Употреба на абстрактни класове}
\begin{lstlisting}
public  abstract class BankAccount {
 	protected int balance;

 	public int getBalance() {
 		return balance;
	}

 	public void deposit(int amount) {
 		balance = balance + amount;
	}

 	public abstract void withdraw(int amount);
}
\end{lstlisting}
\end{frame}

\begin{frame}[containsverbatim]\frametitle{Пример: Наследяване на абстрактни класове}
\begin{lstlisting}
public  class CheckingAccount extends BankAccount {
 	public CheckingAccount () {
 		balance = 0;
	}

 	public void withdraw (int amount) {
 		balance = balance - amount;
	}
}
\end{lstlisting}
\end{frame}

\begin{frame}[containsverbatim]\frametitle{Пример: Наследяване на абстрактни класове}
\begin{lstlisting}
public  class SavingsAccount extends BankAccount {
 	private int numberOfWithdrawals;
 	public SavingsAccount() {
 		balance = 0;
 		numberOfWithdrawals = 0;
	}
 	public void withdraw(int amount) {
 		if (numberOfWithdrawals > 5) {
 			throw new RuntimeException(
				"Cannot make >5 withdrawals a month");
 		} else {
 			balance = balance - amount;
 			numberOfWithdrawals++;
		}
	}
 	public void resetNumOfWithdrawals() {
		numberOfWithdrawals=0;
	}
}
\end{lstlisting}
\end{frame}

\end{document}
