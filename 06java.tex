\PassOptionsToPackage{dvipsnames}{xcolor}
\documentclass[ignorenonframetext, hyperref=unicode,compress,pdflatex]{beamer}
%\documentclass[ignorenonframetext, hyperref=unicode,pdftex]{beamer}


\usepackage{cmap}
\usepackage[T2A]{fontenc}
\usepackage[utf8x]{inputenc}
\usepackage[bulgarian]{babel}
\selectlanguage{bulgarian}

\usepackage{graphicx}
\usepackage{listings}

\hypersetup{
	colorlinks=true,
	linkcolor=blue,
	filecolor=blue,
	urlcolor=blue,
	anchorcolor=blue,
	citecolor=blue
}

\lstset{language=java, 
  numbers=left, 
  numberstyle=\tiny,
  stepnumber=1, 
  numbersep=3pt, 
  tabsize=2, 
  texcl,
  basicstyle=\ttfamily\small,
  identifierstyle=\ttfamily\small,
  keywordstyle=\sffamily\bfseries\small,
  extendedchars=true, inputencoding=utf8x,
  backgroundcolor=\color[rgb]{1,1,0.845},
  escapeinside={/*@}{@*/}}

\newcommand{\Cpp}{{\ttfamily\bfseries C++}}
\newcommand{\CC}{{\ttfamily\bfseries C}}

% \usepackage{algpseudocode}

%\usepackage{ucs}

%\logo{\includegraphics[height=0.4cm]{../macros/logo_elsys.png}}

\titlegraphic{\href{http://creativecommons.org/licenses/by-sa/3.0/}{\includegraphics{pics/cc.png}}}

\newcommand{\lubo}{%
\author[Л.~Чорбаджиев]{Ненко Табаков, Пламен Танов, Любомир Чорбаджиев}
\institute[ELSYS] {%
Технологично училище ``Електронни системи'' \\
Технически университет, София
}}

\usepackage{tikz}
\usetikzlibrary{arrows,shapes.misc,calc,backgrounds,fit,decorations.pathmorphing,positioning,chains,scopes,topaths,automata,shapes,trees}

\tikzset{obj/.style={rectangle,minimum size=6mm,
				very thick,draw=red!50!black!50,top color=white,bottom color=red!50!black!20, font=\ttfamily},
ref/.style={rectangle,
		minimum size=8mm,
		% The rest
		very thick,draw=black!50,
		top color=white,bottom color=black!20},
	every node/.style={font=\ttfamily},
	>=stealth',
	thick,
	draw=black!50,
	shorten >=1pt,node distance=2cm and 2.5cm,on grid}


\mode<article>
{

}

\mode<presentation>
{
  \usetheme[secheader=true]{Madrid}
  \usecolortheme{crane}
  \usefonttheme[onlylarge]{structurebold}
  \setbeamercovered{transparent}
}

\usepackage[unicode]{hyperref}

%%% Local Variables: 
%%% mode: latex
%%% TeX-master: t
%%% End: 


\title{Интерфейси, абстрактни класове, изключения и вътрешни класове}
\authors
\date{\today}

\usetikzlibrary{shapes}

\bibstyle{plain}

\begin{document}

\frame{\titlepage}


\begin{frame}
\small
{\bf Забележка:} Тази лекция е адаптация на:
\begin{itemize}
  \item \href{http://ocw.mit.edu/NR/rdonlyres/Electrical-Engineering-and-Computer-Science/6-092January--IAP--2006/13D676CC-7269-4406-A215-53245A12B235/0/lecture4.pdf}{Lucy Mendel: {\em Interfaces, Abstract classes, Exceptions, Inner classes}} from
\href{http://ocw.mit.edu/OcwWeb/Electrical-Engineering-and-Computer-Science/6-092January--IAP--2006/CourseHome/index.htm}{
{\em 6.092: Java for 6.170} (MIT OpenCourseWare:
Massachusetts Institute of Technology)}\\
{\bf Лиценз:}
\href{http://ocw.mit.edu/OcwWeb/web/terms/terms/index.htm\#cc}{Creative commons
BY-NC-SA}
\end{itemize}

\end{frame}

\begin{frame}

{\bf Допълнителна литература}
\nocite{*}
\bibliographystyle{plain}
\bibliography{learning-java}

\end{frame}
\begin{frame}
\frametitle{Съдържание}
\tableofcontents %[hideallsubsections]
\end{frame}


\section{Абстрактни класове}

\begin{frame}[containsverbatim]\frametitle{Абстракни класове}
\begin{itemize}
\item Дефинират се с ключовата дума abstract
\item Не могат да се създават обекти от абстрактен клас
\item Използват се, когато част от кода на два класа съвпада
\end{itemize}
\begin{lstlisting}
public abstract class Person {
 	private String name = "";

 	public String getName() {
 		return name;
	}
 	public void setName(String n) {
 		name = n;
	}
 	abstract public String sayGreeting();
}
\end{lstlisting}
\end{frame}


\begin{frame}[containsverbatim]\frametitle{Абстракни класове}
\begin{lstlisting}
class EnglishPerson extends Person { 
 	public String sayGreeting() {
 		return "Hello!";
	}
}

class SpanishPerson extends Person { 
 	public String sayGreeting() {
 		return "Hola!";
	}
}
\end{lstlisting}
\end{frame}


\section{Интерфейси}

\begin{frame}[containsverbatim]\frametitle{Интрефейси}
\begin{itemize}
\item Интерфейсът дава списък на достъпните методи
\item Дефинират се с ключовата дума \lstinline{interface}
\item Реализират се с ключовата дума \lstinline{implements}
\item Използват се за описание на начина на взаимодействие между различни или неизвестни компоненти
\end{itemize}
\begin{lstlisting}
public interface Dragable { 
 	public void drag();
}
public class Icon implements Dragable { 
 	public void drag() { ... }
}
public class Chair implements Dragable {
 	public void drag() { ... }
}
\end{lstlisting}
\end{frame}

\begin{frame}[containsverbatim]\frametitle{Реализация на множество интерфейси}
\begin{itemize}
\item Всеки един клас може да реализира произволен брой (множество) интерфейси
\end{itemize}
\begin{lstlisting}
interface Drawable {
 	public void draw();
}

interface Clickable {
 	public void click();
}

interface Draggable {
 	public void drag();
}
\end{lstlisting}
\end{frame}


\begin{frame}[containsverbatim]\frametitle{Реализация на множество интерфейси}
\begin{lstlisting}
class Icon implements Drawable, Clickable, Draggable {
 	public void draw() {
 		System.out.println("drawing...");
	}
 	public void click() {
 		System.out.println("clicking...");
	}
 	public void drag() {
 		System.out.println("dragging...");
	}
}
\end{lstlisting}
\end{frame}



\section{Наследяване}

\begin{frame}[containsverbatim]\frametitle{Единично наследяване}
\begin{itemize}
\item Всеки един клас може да наследява само един клас
\item Ако той не е указан непосредствено, то родител е \lstinline{java.lang.Object}
\item Възможно е предефиниране на методи
\end{itemize}
\begin{lstlisting}
class Parent { 
 	public String doSomething() {
 		return "Hello (Parent)";
	}
}
class Child extends Parent { 
 	public String doSomething() {
 		return "Hello (Child)";
	}
}
\end{lstlisting}
\end{frame}

\begin{frame}[containsverbatim]\frametitle{Подтипове}
\begin{lstlisting}
class Square{ 
	public int width;
}
class Rectangle{ 
	public int width, height;
}

int calculateArea (Square x) {
    return (x.width)*(x.width); 
}
int calculateCircumference (Rectangle x) {
    return 2*(x.width+x.height); 
}
\end{lstlisting}
\begin{itemize}
\item Дали \lstinline{Square} трябва да наследи \lstinline{Rectangle}, или обратното \lstinline{Rectangle} трябва да наследи \lstinline{Square}?
\end{itemize}
\end{frame}

\begin{frame}[containsverbatim]\frametitle{Подтипове: \lstinline{Rectangle} наследява \lstinline{Square}}
\begin{itemize}
\item Дали \lstinline{Square} трябва да наследи \lstinline{Rectangle}, или обратното \lstinline{Rectangle} трябва да наследи \lstinline{Square}?
\end{itemize}
\begin{lstlisting}
class Square {
   public int width;
   Square( int x ) { width = x; }
}

class Rectangle extends Square {
   public int height;
   Rectangle( int width, int height ) {
        super(width);
        this.height = height; }
}
...
Rectangle rect = new Rectangle( 2, 3 );
calculateArea( rect ) ;     // returns 4, not 6 !
\end{lstlisting}
\end{frame}

\begin{frame}[containsverbatim]\frametitle{Подтипове: \lstinline{Square} наследява \lstinline{Rectangle}}
\begin{itemize}
\item Дали \lstinline{Square} трябва да наследи \lstinline{Rectangle}, или обратното \lstinline{Rectangle} трябва да наследи \lstinline{Square}?
\end{itemize}
\begin{lstlisting}
class Rectangle { 
   public int width, height;
}
class Square extends Rectangle {
    public int side;
}
...
Square square = new Square( 3 );
calculateCircumference( sq ) ; // w.t.f. no height!
\end{lstlisting}
\end{frame}

\begin{frame}[containsverbatim]\frametitle{Подтипове: отново \lstinline{Square} наследява \lstinline{Rectangle}}
\begin{itemize}
\item Дали \lstinline{Square} трябва да наследи \lstinline{Rectangle}, или обратното \lstinline{Rectangle} трябва да наследи \lstinline{Square}?
\end{itemize}
\begin{lstlisting}
class Rectangle {
   public int width, height;
   Rectangle( int width, int height ) {
      this.width = width; this.height = height; }
}
class Square extends Rectangle {
    Square( int x ) { super( x, x ); }
}
...
Square square = new Square( 3 );
calculateCircumference( sq ) ; // 12, ok
\end{lstlisting}
\end{frame}

\begin{frame}[containsverbatim]\frametitle{Подтипове}
\begin{itemize}
\item Наследяването позволява повторно изплозване на кода на родителския клас в класа-наследник
\item Класът наследник се превръща в подтип на базовия клас. За да бъде истински подтип наследникът трябва да се държи коректно, когато се използва от методи, които очакват екземпляр на базовия клас.
\item При наследяване винаги трябва да се осигури правилно функциониране на наследника, когато той замества екземпляр на базовия клас
\end{itemize}
\end{frame}

\begin{frame}[containsverbatim]\frametitle{Наследяване срещу композиция}
\begin{lstlisting}
public class ListSet extends ArrayList { 
...
} 
\end{lstlisting}

\begin{lstlisting}
public class ListSet { // might want to implement Set
  private List myList = new ArrayList();
  public void add(Object o) {
      if (!myList.contains(o)) myList.add(o);
  }
...
}
\end{lstlisting}
\end{frame}


\section{Изключения}

\begin{frame}[containsverbatim]\frametitle{Изключения}
\begin{itemize}
\item Помагат за обработката на възникнали изключителни ситуации
\item Изключението не може просто да бъде изпуснато
\item При възникване на изключения се прекъсва нормалното изпълнение на програмата и се търси код, който е предназначен за обработка на възникналото изключение
\end{itemize}
\begin{lstlisting}
	try {
		// statement(s) that might throw exception
	} catch (ExceptiontypeA name) {
		// handle or report exceptiontypeA
	} catch (ExceptiontypeB name) {
		// handle or report exceptiontypeB
	} finally {
		// clean-up statement(s)
	}
\end{lstlisting}
\end{frame}

\begin{frame}[containsverbatim]\frametitle{Пример: Обработка на изключения}
\begin{lstlisting}
class Editor {
	boolean fileOpen = false;
 	public boolean openFile(String filename) {
 		try {
			fileOpen = true;
 			File f = new File(filename);
 			// действия с f
			...
 			return true;
 		} catch (FileNotFoundException e) {
 			// изпълнява се само при генериране на изключение
			e.printStackTrace();
 			return false;
 		} finally { // изпълнява се винаги:
 			fileOpen = false;
		}
	}
}
\end{lstlisting}
\end{frame}

\begin{frame}[containsverbatim]\frametitle{Генериране на изключения}
\begin{itemize}
\item Използва се ключовата дума \lstinline{throw}, следвана от обект, който да съхранява информация за причината за възникване
\item При дефиницията на конструктор или метод се описват изключенията, които могат да възникнат при изпълнение посредством ключовата дума \lstinline{throws}
\end{itemize}
\begin{lstlisting}
public class File {
 	public File(String filename) 
				throws FileNotFoundException {
		...
 		// файлът не съществува
 		if ( /*  file not found */ ) {
 			throw new FileNotFoundException();
		}
		...
	}
}
\end{lstlisting}
\end{frame}


\section{Вложени класове}

\begin{frame}[containsverbatim]\frametitle{Вложени класове}
\begin{lstlisting}
public class EnclosingClass {
	...
	public class ANestedClass {
		...
	}
	...
}
\end{lstlisting}
\begin{itemize}
\item Кога и защо се използват вложени класове?
\end{itemize}

\end{frame}


\begin{frame}[containsverbatim]\frametitle{Вложени класове}
\begin{itemize}
\item Имат достъп до всички полета на външния клас (дори \lstinline{private} полетата)
\item Могат да бъдат статични (също така \lstinline{final}, \lstinline{abstract}). Когато вътрешният клас е статичен, той има достъп само до статичните полета на външният клас
\item Не-статичните вложени класове се наричат още и вътрешни класове (inner classes)
\end{itemize}
\begin{lstlisting}
class  EnclosingClass {
 	static class StaticNestedClass {
		// ...
	}
 	class InnerClass {
		// ...
	}
}
\end{lstlisting}
\end{frame}

\begin{frame}[containsverbatim]\frametitle{Вътрешни класове}
\begin{itemize}
\item Има достъп до всичките полета на съдържащия го клас
\item Не могат да имат статични полета
\item Не може да съществува без екземпляр на съдържащия го клас
\end{itemize}

\end{frame}

\begin{frame}[containsverbatim]\frametitle{Локални вътрешни класове}
\begin{lstlisting}
import  java.util.ArrayList;
import  java.util.Iterator;
public class Stack {
 	private ArrayList items;
 	public Iterator iterator() {
		// клас дефиниран в тялото на метода
 		class StackIterator implements Iterator {
 			int currentItem = items.size() - 1;
 			public boolean hasNext() { /* ... */ }
 			public ArrayList<Object> next() { /* ... */ }
 			public void remove() { /* ... */ }
		}
		// видим е само в границите на самата функция
 		return new StackIterator();
	}
}
\end{lstlisting}
\end{frame}

\begin{frame}[containsverbatim]\frametitle{Анонимни вътрешни класове}
\begin{lstlisting}
import  java.util.ArrayList;
import  java.util.Iterator;
public class Stack {
 	private ArrayList items;

 	public Iterator iterator() {
 		return new Iterator() {//реализира интерфейса Iterator
 			int currentItem = items.size() - 1;
 			public boolean hasNext() { /* ... */ }
 			public ArrayList<Object> next() { /* ... */ }
 			public void remove() { /* ... */ }
		};
	}
}
\end{lstlisting}
\end{frame}

\section{Вътрешни класове}

\begin{frame}[containsverbatim]\frametitle{Вътрешни класове}
\begin{lstlisting}
public class Animal {
	class Brain {
		// ...
	}
	void performBehavior() {
		Brain brain=new Brain();
		// ...
	}
}
\end{lstlisting}

\begin{lstlisting}
Animal monkey = new Animal( );
Animal.Brain monkeyBrain = monkey.new Brain( );
\end{lstlisting}
\end{frame}


\begin{frame}[containsverbatim]\frametitle{Вътрешни класове като адаптери}
\begin{lstlisting}
public class EmployeeList {
	private Employee[] employees = new Employee[0];
	// ...
	void add(Employee e) {...}
	void remove(int index) {...}
	//...
}
\end{lstlisting}
\begin{lstlisting}
public interface Iterator {
	boolean hasNext();
	Object next();
	void remove();
}
class EmployeeListIterator implements Iterator {
	// Трябва да познава вътрешното устройство на EmployeeList
	//...
}
\end{lstlisting}
\end{frame}

\begin{frame}[containsverbatim]\frametitle{Вътрешни класове като адаптери}
\begin{lstlisting}
public class EmployeeList {
    private Employee[] employees = new Employee[0];
    //...
    class Iterator implements java.util.Iterator {
        private int element = 0;
        public boolean hasNext( ) {
            return  element < employees.length ;
        }
        public Object next( ) {
            if ( hasNext( ) )
                return employees[ element++ ];
            else
                throw new NoSuchElementException( );
        }
        //...
\end{lstlisting}
\end{frame}

\begin{frame}[containsverbatim]\frametitle{Вътрешни класове като адаптери}
\begin{lstlisting}
        public void remove( ) {
             throw new UnsupportedOperationException( );
        }
    }
    //...

    Iterator getIterator() {
        return new Iterator();
    }
}
\end{lstlisting}
\end{frame}


\begin{frame}[containsverbatim]\frametitle{Локални класове}
\begin{lstlisting}
public class Animal {
	public void performBehavior() {
		class Brain {
			// ...
		}
		//...
	}
}
\end{lstlisting}
\end{frame}

\begin{frame}[containsverbatim]\frametitle{Специфики на локалните класове}
\begin{lstlisting}
public class Animal {
	public void performBehavior(final boolean nocturnal){
		class Brain {
			...
			void sleep(){
				if(nocturnal) {
					...
				} else {
					...
				}
			}
		}
		...
	}
}
\end{lstlisting}
\end{frame}

\begin{frame}[containsverbatim]\frametitle{Анонимни класове}
\begin{lstlisting}
Iterator getIterator() {
	return new Iterator() {
		int element = 0;
		public boolean hasNext() {
			return element < employees.length;
		}
		public Object next() {
			if (hasNext())
				return employees[element++];
			else
				throw new NoSuchElementException();
		}
		public void remove() {
			throw new UnsupportedOperationException();
		}
	};
}
\end{lstlisting}
\end{frame}

\begin{frame}[containsverbatim]\frametitle{Област на видимост на \lstinline{this}}
\begin{lstlisting}
class Animal {
	int size;
	class Brain {
		int size;
	}
}
\end{lstlisting}
\begin{lstlisting}
class Brain {
	Animal ourAnimal = Animal.this;
	int animalSize = Animal.this.size;
}
\end{lstlisting}
\end{frame}


\end{document}
