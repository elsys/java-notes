\PassOptionsToPackage{dvipsnames}{xcolor}
\documentclass[ignorenonframetext, hyperref=unicode,compress,pdflatex]{beamer}
%\documentclass[ignorenonframetext, hyperref=unicode,pdftex]{beamer}


\usepackage{cmap}
\usepackage[T2A]{fontenc}
\usepackage[utf8x]{inputenc}
\usepackage[bulgarian]{babel}
\selectlanguage{bulgarian}

\usepackage{graphicx}
\usepackage{listings}

\hypersetup{
	colorlinks=true,
	linkcolor=blue,
	filecolor=blue,
	urlcolor=blue,
	anchorcolor=blue,
	citecolor=blue
}

\lstset{language=java, 
  numbers=left, 
  numberstyle=\tiny,
  stepnumber=1, 
  numbersep=3pt, 
  tabsize=2, 
  texcl,
  basicstyle=\ttfamily\small,
  identifierstyle=\ttfamily\small,
  keywordstyle=\sffamily\bfseries\small,
  extendedchars=true, inputencoding=utf8x,
  backgroundcolor=\color[rgb]{1,1,0.845},
  escapeinside={/*@}{@*/}}

\newcommand{\Cpp}{{\ttfamily\bfseries C++}}
\newcommand{\CC}{{\ttfamily\bfseries C}}

% \usepackage{algpseudocode}

%\usepackage{ucs}

%\logo{\includegraphics[height=0.4cm]{../macros/logo_elsys.png}}

\titlegraphic{\href{http://creativecommons.org/licenses/by-sa/3.0/}{\includegraphics{pics/cc.png}}}

\newcommand{\lubo}{%
\author[Л.~Чорбаджиев]{Ненко Табаков, Пламен Танов, Любомир Чорбаджиев}
\institute[ELSYS] {%
Технологично училище ``Електронни системи'' \\
Технически университет, София
}}

\usepackage{tikz}
\usetikzlibrary{arrows,shapes.misc,calc,backgrounds,fit,decorations.pathmorphing,positioning,chains,scopes,topaths,automata,shapes,trees}

\tikzset{obj/.style={rectangle,minimum size=6mm,
				very thick,draw=red!50!black!50,top color=white,bottom color=red!50!black!20, font=\ttfamily},
ref/.style={rectangle,
		minimum size=8mm,
		% The rest
		very thick,draw=black!50,
		top color=white,bottom color=black!20},
	every node/.style={font=\ttfamily},
	>=stealth',
	thick,
	draw=black!50,
	shorten >=1pt,node distance=2cm and 2.5cm,on grid}


\mode<article>
{

}

\mode<presentation>
{
  \usetheme[secheader=true]{Madrid}
  \usecolortheme{crane}
  \usefonttheme[onlylarge]{structurebold}
  \setbeamercovered{transparent}
}

\usepackage[unicode]{hyperref}

%%% Local Variables: 
%%% mode: latex
%%% TeX-master: t
%%% End: 


\title{Java 8: ламбда функции и Stream API}
\authors
\date{\today}

\usetikzlibrary{shapes}

\bibstyle{plain}

\begin{document}

\frame{\titlepage}


\begin{frame}
\frametitle{Съдържание}
\tableofcontents
\end{frame}


\section{Ламбда функции}

\begin{frame}[containsverbatim]\frametitle{Пример: филтриране на списък от хора}
\begin{itemize}
\item Дефиниция на класа \lstinline{Person}:
\end{itemize}
\begin{lstlisting}
public class Person {

	private String names;

	private Gender gender;

	private int age;

    // all args constructor \& getters...
}
\end{lstlisting}
\begin{itemize}
\item Как ще изглежда функция, която филтрира в един списък хората на възраст
под 25 години?
\end{itemize}
\end{frame}


\begin{frame}[containsverbatim]\frametitle{Пример: филтриране на списък от хора}
\begin{itemize}
\item Решение:
\end{itemize}
\begin{lstlisting}
public static List<Person> filterByAgeLessThan(
		List<Person> people, int age) {
	List<Person> result = new ArrayList<>();
	for (Person person : people) {
		if (person.getAge() < age) {
			result.add(person);
		}
	}
	return result;
}
\end{lstlisting}
\begin{itemize}
\item А как ще изглежда функция, която филтрира хората на определена възраст и
от определен пол?
\end{itemize}
\end{frame}

\begin{frame}[containsverbatim]\frametitle{Пример: филтриране на списък от хора}
\begin{itemize}
  \item Функция филтрираща по още един критерий:
\end{itemize}
\begin{lstlisting}
public static List<Person> filterByAgeAndGender(
		List<Person> people, int age, Gender gender) {
	List<Person> result = new ArrayList<>();
	for (Person person : people) {
		if (person.getAge() == age
				&& person.getGender() == gender) {
			result.add(person);
		}
	}
	return result;
}
\end{lstlisting}
\begin{itemize}
  \item Какви са разликите с предишната функция? Как може еднаквите части да се
  изнесат на едно място?
\end{itemize}
\end{frame}

\begin{frame}[containsverbatim]\frametitle{Решение преди Java 8}
\begin{itemize}
  \item Добавя се параметър на филтриращата функция от абстрактен тип, в чиито
  абстрактни методи да се дефинира при извикване поведението, което искаме да се
  променя динамично
\end{itemize}
\begin{lstlisting}
public interface Filter {
	boolean matches(Person person);
}
\end{lstlisting}
\begin{lstlisting}
public static List<Person> filter(List<Person> people,
		Filter filter) {
	List<Person> result = new ArrayList<>();
	for (Person person : people) {
		if (filter.matches(person)) {
			result.add(person);
		}
	}
	return result;
}
\end{lstlisting}
\end{frame}

\begin{frame}[containsverbatim]\frametitle{Решение преди Java 8}
\begin{itemize}
  \item Извикването на функцията обикновено става с
  инстанциирането на анонимен клас:
\end{itemize}
\begin{lstlisting}
List<Person> people = Arrays.asList(
		new Person("Ivan", Gender.MALE, 22),
		new Person("Ivanka", Gender.FEMALE, 34));

List<Person> result = filter(people, new Filter() {

	public boolean matches(Person person) {
		return person.getAge() > 12 && person.getAge() < 65
				&& person.getGender() == Gender.FEMALE;
	}
}); // ще съдържа само инстанцията с име "Ivanka"
\end{lstlisting}
\begin{itemize}
  \item Но този синтаксис е неудобен, дълъг и по-трудно четим\ldots
\end{itemize}
\end{frame}

\begin{frame}[containsverbatim]\frametitle{Java 8 синтаксис}
\begin{itemize}
  \item Затова в Java 8 той е опростен чрез т. нар. ламбда функции - функции
  подадени като аргумент на други функции.
\end{itemize}
\begin{lstlisting}
List<Person> result = filter(people,
		person -> person.getAge() > 12
				&& person.getAge() < 65
				&& person.getGender() == Gender.FEMALE)
}); // ще съдържа само инстанцията с име "Ivanka"
\end{lstlisting}
\begin{itemize}
  \item Декларацията на функцията \lstinline{filter} от примера остава същата
  \item Резултатът от операцията в ламбда функцията се връща като резултат при
  извикването на \lstinline{Filter::matches} в имплементацията на
  \lstinline{filter}
\end{itemize}
\end{frame}

\begin{frame}[containsverbatim]\frametitle{Функционални интерфейси}
\begin{itemize}
  \item Интерфейсът \lstinline{Filter} наричаме функционален
  \item Функционален интерфейс е този, в който има точно един абстрактен метод
  \item За да може да се прилага ламбда синтаксисът, типът на аргумента, на
  който се подава ламбда функцията, трябва да бъде "функционален интерфейс"
  \item При абстрактни класове (дори и само с един абстрактен метод) или
  интерфейси с повече от един абстрактен метод, този синтаксис не може да се
  използва (в тези случаи могат да се използват анонимни класове)
\end{itemize}
\end{frame}

\begin{frame}[containsverbatim]\frametitle{Функционални интерфейси}
\begin{itemize}
  \item В стандартната библиотека в пакета \lstinline{java.util.function} има
  дефинирани множество шаблонни функционални интерфейси, които се използват както в самата библиотеката, така
  и могат да се използват от всеки програмист.
  \item \lstinline{Function} - дефинира метод, който приема един шаблонен
  аргумент и връща шаблонен резултат
  \item \lstinline{Consumer} - метод с един шаблонен
  аргумент и връщащ \lstinline{void}
  \item \lstinline{Supplier} - метод без аргументи и връщащ
  шаблонен резултат
  \item \lstinline{Predicate} - метод с един шаблонен
  аргумент и връщащ резултат от тип \lstinline{boolean}
  \item Съществуват вариации на тези функционални методи, които дефинират същите
  методи, но с два аргумента и се казват съответно \lstinline{BiFunction},
  \lstinline{BiConsumer} и т.н.
\end{itemize}
\end{frame}

\begin{frame}[containsverbatim]\frametitle{Пример: сортиране на списък с
\lstinline{Comparator}}
\begin{itemize}
  \item В \lstinline{java.util.List} има метод \lstinline{sort} с единствен
  аргумент от тип \lstinline{Comparator}, който е функционален интерфейс
\end{itemize}
\begin{lstlisting}
public interface Comparator<T> {
	int compare(T o1, T o2);
}
\end{lstlisting}
\begin{lstlisting}
List<Person> people = Arrays.asList(
		new Person("Ivan", Gender.MALE, 22),
		new Person("Ivanka", Gender.FEMALE, 34),
		new Person("Peter", Gender.FEMALE, 11));
people.sort((p1, p2) -> p1.getAge() - p2.getAge());
\end{lstlisting}
\end{frame}

\begin{frame}[containsverbatim]\frametitle{Ламбда функция на много редове}
\begin{itemize}
  \item В предишните примери ламбда функциите се състояха от само един израз
  \item Понякога обаче е необходимо да се извършат няколко операции в тях преди
  да се върне резултат
  \item Тогава тялото на ламбда функцията е оградено с
  \lstinline[mathescape]!{}! и задължително има \lstinline{return} израз, когато
  се очаква функцията да върне резултат
  \item Следващият пример е еквивалентен на предишния:
\end{itemize}
\begin{lstlisting}
people.sort((p1, p2) -> {
	int delta = p1.getAge() - p2.getAge();
	return delta;
});
\end{lstlisting}
\end{frame}

\begin{frame}[containsverbatim]\frametitle{Ламбда функция на много редове}
\begin{itemize}
  \item Методът \lstinline{forEach} в \lstinline{java.util.List} може да се
  използва за обхождане на всички елементи
  \item Той приема като аргумент \lstinline{Consumer} и затова не очаква
  \lstinline{return} израз, дори и когато ламбда функцията е на много редове
\end{itemize}
\begin{lstlisting}
people.forEach(person -> {
	if (person.getAge() > 65) {
		System.out.printf("%s, 65+\n", person.getNames());
	} else {
		System.out.printf("%s, %d\n", person.getNames(),
				person.getAge());
	}
});
\end{lstlisting}
\end{frame}

\section{Stream API}

\begin{frame}[containsverbatim]\frametitle{Необходимостта от Stream API}
\begin{itemize}
  \item Как ще изглежда програма, която намира списък с имената на всички жени
  под 25 години, подреден по възрастта им?
\end{itemize}
\begin{lstlisting}
List<Person> womenUnder25 = new ArrayList<>();
for (Person person : people) {
	if (person.getGender() == Gender.FEMALE
			&& person.getAge() < 25) {
		womenUnder25.add(person);
	}
}
Collections.sort(womenUnder25,
		(p1, p2) -> p1.getAge() - p2.getAge());
List<String> womenUnder25Names = new ArrayList<>();
for (Person person : people) {
	womenUnder25Names.add(person.getNames());
}
\end{lstlisting}
\end{frame}

\begin{frame}[containsverbatim]\frametitle{Необходимостта от Stream API}
\begin{itemize}
  \item Решенията на различни задачи за филтриране, трансформиране на списък с
  обекти, редуцирането му до единствена стойност и други са с много подобен код
  \item Установяването на целите му, от човек, който не го е виждал, няма да е
  лесно и бързо
  \item Не можем ли просто да декларираме какво искаме да се случи и итерациите
  по списъка да бъдат скрити като имплементационен детайл, който не ни
  интересува?
\end{itemize}
\end{frame}

\begin{frame}[containsverbatim]\frametitle{Необходимостта от Stream API}
\begin{itemize}
  \item Решение, използвайки добавения в Java 8 Stream API:
\end{itemize}
\begin{lstlisting}
List<String> womenUnder25Names = people.stream()
	.filter(person -> person.getGender() == Gender.FEMALE)
	.filter(person -> person.getAge() < 25)
	.sorted((p1, p2) -> p1.getAge() - p2.getAge())
	.map(person -> person.getNames())
	.collect(Collectors.toList());
\end{lstlisting}
\end{frame}

\begin{frame}[containsverbatim]\frametitle{Какво е Stream? Stream vs Collection}
\begin{itemize}
  \item \textbf{Поредица от елементи} - подобно на колекцията, потокът
  предоставя интерфейс към поредица от елементи от определен тип, но за
  разлика от нея не се фокусира върху запазването и достъпа им от паметта, а
  върху изчисленията, които трябва се извършат с тях
  \item \textbf{Базиран на източник на данни} - често това е колекция, масив или
  I/O ресурс, като този източник може да бъде безкраен
  \item \textbf{Композиращ операции върху данните} - поддържат се различни
  операции като \lstinline{filter}, \lstinline{map}, \lstinline{reduce},
  \lstinline{sort} и други, които могат да се композират в различен ред
  взависимост от целта
\end{itemize}
\end{frame}

\begin{frame}[containsverbatim]\frametitle{Крайни и междинни операции}
\begin{lstlisting}
people.stream()
		.filter(person -> person.getAge() < 25)
		.map(person -> person.getNames())
		.collect(Collectors.toList());
\end{lstlisting}
\begin{itemize}
  \item Извиквайки новият за \lstinline{List} метод \lstinline{stream} създаваме
  от списъка обект от тип \lstinline{Stream}
  \item Операциите \lstinline{filter} и \lstinline{map} са \textbf{междинни}
  \item Всяка от междинните операции връща нов обект от тип \lstinline{Stream} и
  това позволява лесното им композиране една след друга
  \item Когато композираме междинни операции, никоя от тях не се изпълнява,
  докато не извикаме \textbf{крайна} операция
  \item Операцията \lstinline{collect} е \textbf{крайна} и трансформира потока в
  списък
  \item Крайна операция върху поток може да се изпълни само веднъж и нейното
  извикване кара всички междинни операции да се изпълнят
\end{itemize}
\end{frame}

\begin{frame}[containsverbatim]\frametitle{Крайни и междинни операции (примери)}
\begin{itemize}
  \item Следният код няма да доведе до изпълнение на операциите
  \lstinline{filter} и \lstinline{map} върху данните от списъка
  \lstinline{people}, тъй като липсва крайна операция
\end{itemize}
\begin{lstlisting}
Stream<String> stream = people.stream()
		.filter(person -> person.getAge() < 25)
		.map(person -> person.getNames());
\end{lstlisting}
\end{frame}

\begin{frame}[containsverbatim]\frametitle{Крайни и междинни операции (примери)}
\begin{itemize}
  \item Следният пример ще доведе до
  \lstinline{java.lang.IllegalStateException}, защото върху един поток са
  изпълнени две крайни операции - \lstinline{collect} и \lstinline{forEach}
\end{itemize}
\begin{lstlisting}
Stream<String> stream = people.stream()
		.filter(person -> person.getAge() < 25)
		.map(person -> person.getNames());
stream.collect(Collectors.toList());
stream.forEach(name -> System.out.println(name));\end{lstlisting}
\end{frame}

\begin{frame}[containsverbatim]\frametitle{Mеждинни операции:
\lstinline{filter}}
\begin{itemize}
  \item \lstinline{filter} оставя в потока само елементи, за които условието се
  изчислява до \lstinline{true}
\end{itemize}
\begin{lstlisting}
List<Person> youngPeople = people.stream()
		.filter(person -> person.getAge() < 25)
		.collect(Collectors.toList());
// в youngPeople ще останат само хора под 25 години
\end{lstlisting}
\end{frame}

\begin{frame}[containsverbatim]\frametitle{Mеждинни операции:
\lstinline{map}}
\begin{itemize}
  \item \lstinline{map} трансформира всеки обект в един поток до друг обект
  посредством правилото подадено като ламбда функция
  \item Потокът, който се връща, е от тип \lstinline{Stream<T>}, където
  \lstinline{T} е типът на върнатия от ламбда функцията обект (в случая -
  \lstinline{String})
\end{itemize}
\begin{lstlisting}
List<String> names = people.stream()
		.map(person -> person.getNames())
		.collect(Collectors.toList());
// в names ще останат само имената на хората от people
\end{lstlisting}
\end{frame}

\begin{frame}[containsverbatim]\frametitle{Mеждинни операции:
\lstinline{flatMap}}
\begin{itemize}
  \item За следващия пример, нека добавим в класа \lstinline{Person} поле, което
  да съдържа списък с всички приятели на дадения човек
\end{itemize}
\begin{lstlisting}
class Person {
	private List<Person> friends;
	...
}
\end{lstlisting}
\begin{itemize}
  \item \lstinline{flatMap} се използва, за да трансформираме един поток от
  списъци от елементи в поток от елементи
  \item Задължително е ламбда функцията, подадена на \lstinline{flatMap}, да
  връща поток
\end{itemize}
\begin{lstlisting}
List<Person> friends = people.stream()
		.flatMap(person -> person.getFriends().stream())
		.collect(Collectors.toList());
// friends ще съдържа всички приятели на хората в people
\end{lstlisting}
\end{frame}

\begin{frame}[containsverbatim]\frametitle{Mеждинни операции:
\lstinline{distinct}}
\begin{itemize}
  \item \lstinline{distinct} премахва повтарящите се елементи от потока 
\end{itemize}
\begin{lstlisting}
long uniqueCount = people.stream()
		.map(person -> person.getNames().split(" ")[0])
		.distinct()
		.count();
// uniqueCount ще съдържа броят на уникалните първи имена
\end{lstlisting}
\end{frame}

\begin{frame}[containsverbatim]\frametitle{Mеждинни операции:
\lstinline{sorted}}
\begin{itemize}
  \item \lstinline{sorted} има две версии - тази без аргументи очаква елементите
  в потока да имплементират интерфейса \lstinline{Comparable}
  \item Другата версия приема като аргумент \lstinline{Comparator}
\end{itemize}
\begin{lstlisting}
List<String> names = people.stream()
		.sorted((p1, p2) -> p1.getAge() - p2.getAge())
		.map(person -> person.getNames())
		.collect(Collectors.toList());
// подрежда хората по възраст и връща само имената им
\end{lstlisting}
\end{frame}

\begin{frame}[containsverbatim]\frametitle{Mеждинни операции:
\lstinline{limit} и \lstinline{skip}}
\begin{itemize}
  \item \lstinline{limit} лимитира елементите до подаденото число
  \item \lstinline{skip} премахна първите n елемента от потока 
\end{itemize}
\begin{lstlisting}
List<Person> result = people.stream()
		.skip(1)
		.limit(5)
		.collect(Collectors.toList());
// ще пропусне първия елемент и след това ще добави максимум 
// следващите 5 елемента в result
\end{lstlisting}
\end{frame}

\begin{frame}[containsverbatim]\frametitle{Mеждинни операции:
\lstinline{peek}}
\begin{itemize}
  \item \lstinline{peek} не променя потока, но позволява да се извършват някакви
  операции с елементите в него
  \item Това го прави удобен за ``debug'' цели
\end{itemize}
\begin{lstlisting}
people.stream()
		.map(person -> person.getNames())
		.peek(names -> System.out.println(names))
		.disctinct()
		.count();
// ще изброи хората с уникални имена, като ще изпринтира
// имената на всички хора
\end{lstlisting}
\end{frame}

\begin{frame}[containsverbatim]\frametitle{Крайни операции:
\lstinline{count} и \lstinline{forEach}}
\begin{itemize}
  \item \lstinline{count} ще върне броя на елементите в потока
\end{itemize}
\begin{lstlisting}
long count = people.stream()
		.filter(person -> person.getAge() % 2 == 0)
		.count();
\end{lstlisting}
\begin{itemize}
  \item \lstinline{forEach} ще изпълни подадената ламбда функция за всеки
  елемент в потока
\end{itemize}
\begin{lstlisting}
people.stream()
		.map(person -> person.getNames())
		.forEach(name -> System.out.println(name));
\end{lstlisting}
\end{frame}

\begin{frame}[containsverbatim]\frametitle{Крайни операции:
\lstinline{reduce}}
\begin{itemize}
  \item \lstinline{reduce} ще комбинира елементите чрез подадената ламбда
  функция, докато не остане само един
  \item Като първи аргумент се подава първоначалната стойност, от която да
  започне комбинирането на елементи
\end{itemize}
\begin{lstlisting}
long ageSum = people.stream()
		.map(person -> person.getAge())
		.reduce(0, (l, r) -> l + r);
\end{lstlisting}
\begin{lstlisting}
String namesSeparatedByComma = people.stream()
		.map(person -> person.getNames())
		.reduce("", (l, r) -> l + ", " + r);
\end{lstlisting}
\begin{itemize}
  \item Забележка: \lstinline{reduce} има и още две версии с различни аргументи
\end{itemize}
\end{frame}

\begin{frame}[containsverbatim]\frametitle{Крайни операции:
\lstinline{collect}}
\begin{itemize}
  \item \lstinline{collect} приема като аргумент имплементация на интерфейса
  \lstinline{java.util.stream.Collector}
  \item В класа \lstinline{java.util.stream.Collectors} има дефинирани няколко
  статични метода, които връщат различни имплементации на \lstinline{Collector}
  интерфейса
  \item Всеки програмист може и сам да дефинира своя имплементация, но това е
  извън обхвата на тази лекция
\end{itemize}
\end{frame}

\begin{frame}[containsverbatim]\frametitle{\lstinline{collect}: трансформиране
в колекция}
\begin{itemize}
  \item \lstinline{Collectors.toList()} записва елементите от потока в
  списък
\end{itemize}
\begin{lstlisting}
List<String> nameList = people.stream()
		.map(person -> person.getNames())
		.collect(Collectors.toList());
\end{lstlisting}
\begin{itemize}
  \item \lstinline{Collectors.toSet()} записва елементите от потока в множество
\end{itemize}
\begin{lstlisting}
Set<String> nameSet = people.stream()
		.map(person -> person.getNames())
		.collect(Collectors.toSet());
\end{lstlisting}
\begin{itemize}
  \item \lstinline{Collectors.toCollection(Supplier<C>)} записва елементите в
  колекцията, която се създава от ламбда функцията
\end{itemize}
\begin{lstlisting}
Queue<String> nameQueue = people.stream()
		.map(person -> person.getNames())
		.collect(Collectors.toCollection(() ->
				new SynchronousQueue<>()));
\end{lstlisting}
\end{frame}

\begin{frame}[containsverbatim]\frametitle{\lstinline{collect}: трансформиране
в \lstinline{java.util.Map}}
\begin{itemize}
  \item Чрез \lstinline{Collectors.toMap()} можем да трансформираме елементите
  от потока в двойки ключ-стойност
  \item Първият аргумент е ламбда функция, която връща ключ за всеки обект
  \item Вторият аргумент е ламбда функция, която връща стойност за всеки обект
  \item Третият аргумент е опционален и се използва за разрешаване на конфликти
  (когато има повторение на ключове)
\end{itemize}
\begin{lstlisting}
Map<String, Gender> genderByPerson = people.stream()
		.collect(Collectors.toMap(
				person -> person.getNames(),
				person -> person.getGender(),
				(g1, g2) -> g1));
\end{lstlisting}

\end{frame}

\begin{frame}[containsverbatim]\frametitle{\lstinline{collect}: групиране на
елементите}
\begin{itemize}
  \item \lstinline{Collectors.groupingBy()} може да се използва за групиране на
  елементите по даден критерий
\end{itemize}
\begin{lstlisting}
Map<Gender, List<Person>> byGender = people.stream()
		.collect(Collectors.groupingBy(
				person -> person.getGender()));
\end{lstlisting}
\begin{itemize}
  \item Подобен на този колектор е и \lstinline{Collectors.partitioningBy()},
  който групира елементите в 2 групи - в зависимост от това дали изпълняват, или
  не дадено условие
\end{itemize}
\begin{lstlisting}
Map<Boolean, List<Person>> byAgeLt25 = people.stream()
		.collect(Collectors.partitioningBy(
				person -> person.getAge() < 25));
\end{lstlisting}
\end{frame}

\begin{frame}[containsverbatim]\frametitle{\lstinline{collect}: комбиниране на
колектори}
\begin{itemize}
  \item Проблем: как може да се групират имената на хората по пол? 
  \item За целта \lstinline{Collectors.groupingBy()} и
  \lstinline{Collectors.partitioningBy()} имат версии, които приемат следващ
  колектор като параметър (нарича се downstream колектор)
  \item За решение на проблема използваме колектора
  \lstinline{Collectors.mapping()}, който приема като аргумент ламбда
  функция, която да вземе имената на групираните вече хора, и колектор,
  който да трансформира потока от имена в списък
\end{itemize}
\begin{lstlisting}
Map<Gender, List<String>> byGender = people.stream()
		.collect(Collectors.groupingBy(
				person -> person.getGender(),
				Collectors.mapping(person -> person.getNames(),
						Collectors.toList())));
\end{lstlisting}
\end{frame}

\begin{frame}[containsverbatim]\frametitle{Други методи от
\lstinline{Collectors}}
\begin{itemize}
  \item \lstinline{collectingAndThen(Collector, Function)} - 
  адаптира подадения колектор да изпълнява допълнителна операция за
  трансформиране на елементите
  \item \lstinline{counting()} - връща колектор, който изброява елементите в
  потока
  \item \lstinline{minBy(Comparator)} и \lstinline{maxBy(Comparator)} -
  колекторът намира съответно най-малкия и най-големия елемент в потока,
  използвайки за сравнение подадения \lstinline{Comparator}
  \item \lstinline{summingInt(ToIntFunction)} - колекторът, използвайки
  подадената ламбда функция, трансформира елементите в \lstinline{int} и ги
  събира; има аналогични колектори за \lstinline{long} и \lstinline{double}
  \item \lstinline{averagingInt(ToIntFunction)} - аналогичен колектор, но
  смята средното аритметично; също има за \lstinline{long} и \lstinline{double}
\end{itemize}
\end{frame}

\section{Задачи за упражнение}

\begin{frame}[containsverbatim]\frametitle{Задачи за упражнение}
\begin{itemize}
  \item За всички задачи приемете, че имате списък с хора
  \item Трансформирайте списъка в стринг, в следния формат: ``<names> (<age>),
  \ldots''
  \item Филтрирайте хората, които нямат приятели и пресметнете за останалите
  средната възраст на приятелите им; запишете резултата в масив с ключ имената
  на човека и стойност средната възраст на приятелите му
  \item Пресметнете средната възраст по пол на хората
  \item Пресметнете броят на хората от всеки пол
  \item Групирайте хората първо по пол, а след това по това дали са на повече
  или по-малко от 25 години
  \item Групирайте хората по възраст над/под 25 години, като върнете
  само множество от имената им (тип на резултата:
  \lstinline{Map<Boolean, Set<String>>})
\end{itemize}
\end{frame}


\end{document}
