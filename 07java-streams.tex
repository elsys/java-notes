\PassOptionsToPackage{dvipsnames}{xcolor}
\documentclass[ignorenonframetext, hyperref=unicode,compress,pdflatex]{beamer}
%\documentclass[ignorenonframetext, hyperref=unicode,pdftex]{beamer}


\usepackage{cmap}
\usepackage[T2A]{fontenc}
\usepackage[utf8x]{inputenc}
\usepackage[bulgarian]{babel}
\selectlanguage{bulgarian}

\usepackage{graphicx}
\usepackage{listings}

\hypersetup{
	colorlinks=true,
	linkcolor=blue,
	filecolor=blue,
	urlcolor=blue,
	anchorcolor=blue,
	citecolor=blue
}

\lstset{language=java, 
  numbers=left, 
  numberstyle=\tiny,
  stepnumber=1, 
  numbersep=3pt, 
  tabsize=2, 
  texcl,
  basicstyle=\ttfamily\small,
  identifierstyle=\ttfamily\small,
  keywordstyle=\sffamily\bfseries\small,
  extendedchars=true, inputencoding=utf8x,
  backgroundcolor=\color[rgb]{1,1,0.845},
  escapeinside={/*@}{@*/}}

\newcommand{\Cpp}{{\ttfamily\bfseries C++}}
\newcommand{\CC}{{\ttfamily\bfseries C}}

% \usepackage{algpseudocode}

%\usepackage{ucs}

%\logo{\includegraphics[height=0.4cm]{../macros/logo_elsys.png}}

\titlegraphic{\href{http://creativecommons.org/licenses/by-sa/3.0/}{\includegraphics{pics/cc.png}}}

\newcommand{\lubo}{%
\author[Л.~Чорбаджиев]{Ненко Табаков, Пламен Танов, Любомир Чорбаджиев}
\institute[ELSYS] {%
Технологично училище ``Електронни системи'' \\
Технически университет, София
}}

\usepackage{tikz}
\usetikzlibrary{arrows,shapes.misc,calc,backgrounds,fit,decorations.pathmorphing,positioning,chains,scopes,topaths,automata,shapes,trees}

\tikzset{obj/.style={rectangle,minimum size=6mm,
				very thick,draw=red!50!black!50,top color=white,bottom color=red!50!black!20, font=\ttfamily},
ref/.style={rectangle,
		minimum size=8mm,
		% The rest
		very thick,draw=black!50,
		top color=white,bottom color=black!20},
	every node/.style={font=\ttfamily},
	>=stealth',
	thick,
	draw=black!50,
	shorten >=1pt,node distance=2cm and 2.5cm,on grid}


\mode<article>
{

}

\mode<presentation>
{
  \usetheme[secheader=true]{Madrid}
  \usecolortheme{crane}
  \usefonttheme[onlylarge]{structurebold}
  \setbeamercovered{transparent}
}

\usepackage[unicode]{hyperref}

%%% Local Variables: 
%%% mode: latex
%%% TeX-master: t
%%% End: 


\title{Java 8: ламбда функции и Stream API}
\authors
\date{\today}

\usetikzlibrary{shapes}

\bibstyle{plain}

\begin{document}

\frame{\titlepage}


\begin{frame}
\frametitle{Съдържание}
\tableofcontents %[hideallsubsections]
\end{frame}


\section{Ламбда функции}

\begin{frame}[containsverbatim]\frametitle{Пример: филтриране на списък от хора}
\begin{itemize}
\item Дефиниция на класа \lstinline{Person}:
\end{itemize}
\begin{lstlisting}
public class Person {

	private String names;

	private Gender gender;

	private int age;

    // all args constructor \& getters...
}
\end{lstlisting}
\begin{itemize}
\item Как ще изглежда функция, която филтрира в един списък хората на възраст
под 25 години?
\end{itemize}
\end{frame}


\begin{frame}[containsverbatim]\frametitle{Пример: филтриране на списък от хора}
\begin{itemize}
\item Функция, която филтрира в един списък хората, които са под определена
възраст:
\end{itemize}
\begin{lstlisting}
public static List<Person> filterByAgeLessThan(List<Person> people, int age) {
	List<Person> result = new ArrayList<>();
	for (Person person : people) {
		if (person.getAge() < age) {
			result.add(person);
		}
	}
	return result;
}
\end{lstlisting}
\begin{itemize}
\item А как ще изглежда функция, която филтрира хората на определена възраст и
от определен пол?
\end{itemize}
\end{frame}

\begin{frame}[containsverbatim]\frametitle{Пример: филтриране на списък от хора}
\begin{itemize}
  \item Функция филтрираща по още един критерий:
\end{itemize}
\begin{lstlisting}
public static List<Person> filterByAgeAndGender(List<Person> people, int age, Gender gender) {
	List<Person> result = new ArrayList<>();
	for (Person person : people) {
		if (person.getAge() == age && person.getGender() == gender) {
			result.add(person);
		}
	}
	return result;
}
\end{lstlisting}
\begin{itemize}
  \item Какви са разликите с предишната функция? Как може еднаквите части да се
  изнесат на едно място?
\end{itemize}
\end{frame}

\begin{frame}[containsverbatim]\frametitle{Решение преди Java 8}
\begin{itemize}
  \item Добавя се параметър на филтриращата функция от абстрактен тип, в чиито
  абстрактни методи да се дефинира при извикване поведението, което искаме да се
  променя динамично
\end{itemize}
\begin{lstlisting}
public interface Filter {
	boolean matches(Person person);
}
\end{lstlisting}
\begin{lstlisting}
public static List<Person> filter(List<Person> people,
		Filter filter) {
	List<Person> result = new ArrayList<>();
	for (Person person : people) {
		if (filter.matches(person)) {
			result.add(person);
		}
	}
	return result;
}
\end{lstlisting}
\end{frame}

\begin{frame}[containsverbatim]\frametitle{Решение преди Java 8}
\begin{itemize}
  \item Извикването на функцията обикновено (но не задължително) става с
  инстанциирането на анонимен клас:
\end{itemize}
\begin{lstlisting}
List<Person> people = Arrays.asList(
		new Person("Ivan", Gender.MALE, 22),
		new Person("Ivanka", Gender.FEMALE, 34));

List<Person> result = filter(people, new Filter() {

	public boolean matches(Person person) {
		return person.getAge() > 12 && person.getAge() < 65
				&& person.getGender() == Gender.FEMALE;
	}
}); // ще съдържа само инстанцията с име "Ivanka"
\end{lstlisting}
\begin{itemize}
  \item Но този синтаксис е неудобен, дълъг и по-трудно четим\ldots
\end{itemize}
\end{frame}

\begin{frame}[containsverbatim]\frametitle{Java 8 синтаксис}
\begin{itemize}
  \item Затова в Java 8 той е опростен чрез т. нар. ламбда функции - функции
  подадени като аргумент на други функции.
\end{itemize}
\begin{lstlisting}
List<Person> result = filter(people,
		person -> person.getAge() > 12
				&& person.getAge() < 65
				&& person.getGender() == Gender.FEMALE)
}); // ще съдържа само инстанцията с име "Ivanka"
\end{lstlisting}
\begin{itemize}
  \item Декларацията на функцията \lstinline{filter} от примера остава същата
  \item Резултатът от операцията в ламбда функцията се връща като резултат при
  извикването на \lstinline{matches} (върху параметъра от тип
  \lstinline{Filter}) в имплементацията на \lstinline{filter}
\end{itemize}
\end{frame}

\begin{frame}[containsverbatim]\frametitle{Функционални интерфейси}
\begin{itemize}
  \item Интерфейсът \lstinline{Filter} наричаме функционален
  \item Функционален интерфейс е този, в който има точно един абстрактен метод
  \item За да може да се прилага този синтаксис, аргументът, на който се подава
  ламбда функцията, трябва да бъде инстанция на "функционален интерфейс"
  \item При абстрактни класове (дори и само с един абстрактен метод) или
  интерфейси с повече от един абстрактен метод, този синтаксис не може да се
  използва (в тези случаи могат да се използват анонимни класове)
\end{itemize}
\end{frame}

\begin{frame}[containsverbatim]\frametitle{Функционални интерфейси}
\begin{itemize}
  \item В стандартната библиотека в пакета \lstinline{java.util.function} има
  дефинирани множество шаблонни функционални интерфейси, които се използват както в самата библиотеката, така
  и могат да се използват от всеки програмист.
  \item \lstinline{Function} - дефинира метод, който приема един шаблонен
  аргумент и връща шаблонен резултат
  \item \lstinline{Consumer} - метод с един шаблонен
  аргумент и връщащ \lstinline{void}
  \item \lstinline{Supplier} - метод без аргументи и връщащ
  шаблонен резултат
  \item \lstinline{Predicate} - метод с един шаблонен
  аргумент и връщащ резултат от тип \lstinline{boolean}
  \item Съществуват вариации на тези функционални методи, които дефинират същите
  методи, но с два аргумента и се казват съответно \lstinline{BiFunction},
  \lstinline{BiConsumer} и т.н.
\end{itemize}
\end{frame}

\begin{frame}[containsverbatim]\frametitle{Пример: сортиране на списък с
\lstinline{Comparator}}
\begin{itemize}
  \item В \lstinline{java.util.List} има метод \lstinline{sort} с единствен
  аргумент от тип \lstinline{Comparator}, който е функционален интерфейс
\end{itemize}
\begin{lstlisting}
public interface Comparator<T> {
	int compare(T o1, T o2);
}
\end{lstlisting}
\begin{lstlisting}
List<Person> people = Arrays.asList(
		new Person("Ivan", Gender.MALE, 22),
		new Person("Ivanka", Gender.FEMALE, 34),
		new Person("Peter", Gender.FEMALE, 11));
people.sort((p1, p2) -> p1.getAge() - p2.getAge());
\end{lstlisting}
\end{frame}

\begin{frame}[containsverbatim]\frametitle{Ламбда функция на много редове}
\begin{itemize}
  \item В предишните примери ламбда функциите се състояха от само един израз
  \item Следващият пример използва метода \lstinline{forEach} в
  \lstinline{java.util.List}, за да принтира всички елементи по определен начин
  като използва ламбда функция на много редове
\end{itemize}
\begin{lstlisting}
people.forEach(person -> {
	if (person.getAge() > 65) {
		System.out.printf("%s, 65+\n", person.getNames());
	} else {
		System.out.printf("%s, %d\n", person.getNames(),
				person.getAge());
	}
});
\end{lstlisting}
\end{frame}


\end{document}
